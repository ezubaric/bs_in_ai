
%%%%%%%%%%%%%%%%%%%%%%%%%%%%%%%%%%%%%%%%%%%%%%%%%%%%%%
% DO NOT EDIT THIS FILE, IT IS AUTOMATICALLY GENERATED
% INSTEAD, EDIT THE FILE HERE:
%
% https://docs.google.com/spreadsheets/d/1qRaEKxhyfwjHDWa3burGeqK0zeC00Br8NhJfnMWXJdk/edit?usp=sharing
%
% If you edit this file, your changes will be overwritten from this
% spreadsheet linked above.
%%%%%%%%%%%%%%%%%%%%%%%%%%%%%%%%%%%%%%%%%%%%%%%%%%%%%%%
\begin{enumerate}
\item \textbf{Programming with Purpose I: Data-Centric Computing (CMSC 141)}: This course is an introduction to computing and programming through the lens of data. It aims to give you ways of thinking about solving problems using computation. Students will learn to write programs to process both tabular and structured data, to assess programs both experimentally and theoretically, to apply basic data science concepts, and to discuss big ideas around the communication, use, and social impacts of digital information.
\item \textbf{Data Structures and Algorithms (CMSC 142)}: This course explains the concepts and techniques required to write programs that can handle large amounts of data efficiently. Project-oriented and classroom-tested, it presents a number of important algorithms—supported by motivating examples—that bring meaning to the problems faced by computer programmers. The idea of computational complexity is introduced, demonstrating what can and cannot be computed efficiently at scale, helping programmers make informed judgements about the algorithms they use. 
\end{enumerate}