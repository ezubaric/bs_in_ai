
%%%%%%%%%%%%%%%%%%%%%%%%%%%%%%%%%%%%%%%%%%%%%%%%%%%%%%
% DO NOT EDIT THIS FILE, IT IS AUTOMATICALLY GENERATED
% INSTEAD, EDIT THE FILE HERE:
%
% https://docs.google.com/spreadsheets/d/1qRaEKxhyfwjHDWa3burGeqK0zeC00Br8NhJfnMWXJdk/edit?usp=sharing
%
% If you edit this file, your changes will be overwritten from this
% spreadsheet linked above.
%%%%%%%%%%%%%%%%%%%%%%%%%%%%%%%%%%%%%%%%%%%%%%%%%%%%%%%
\begin{enumerate}
\item \textbf{Programming with Purpose I: Data-Centric Computing (CMSC 141)}: This course is an introduction to computing and programming through the lens of data. It aims to give you ways of thinking about solving problems using computation. Students will learn to write programs to process both tabular and structured data, to assess programs both experimentally and theoretically, to apply basic data science concepts, and to discuss big ideas around the communication, use, and social impacts of digital information.
\item \textbf{Data Structures and Algorithms (CMSC 142)}: This course explains the concepts and techniques required to write programs that can handle large amounts of data efficiently. Project-oriented and classroom-tested, it presents a number of important algorithms—supported by motivating examples—that bring meaning to the problems faced by computer programmers. The idea of computational complexity is introduced, demonstrating what can and cannot be computed efficiently at scale, helping programmers make informed choices about the algorithms they use. 
\item \textbf{Privacy, Security and Ethics for Big Data (INST 366)}: The increasing number of networked information technologies—including internet of things (IoT), wearables, ubiquitous sensing, social sharing platforms, and other AI-driven systems—are generating a tremendous amount of data about individuals, companies, and societies. These technologies offer enormous benefits but also create enormous risks to individual privacy and national security. Further, the ease with which data can be collected from online sources, analyzed, and inferences drawn about individual users raises a wide range of ethical questions about these technologies, their creators, and their users.
\item \textbf{Design and Human Disability and Aging (INST 401 )}: Focuses on the design of consumer products and information systems to enable their use by persons with a wider range of physical, sensory, and cognitive abilities. Overviews aging and major types of impairment as they relate to resulting problems using consumer products and information systems. Focuses on principles of design of mass market products.
\end{enumerate}