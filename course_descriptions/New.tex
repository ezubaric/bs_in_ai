
%%%%%%%%%%%%%%%%%%%%%%%%%%%%%%%%%%%%%%%%%%%%%%%%%%%%%%
% DO NOT EDIT THIS FILE, IT IS AUTOMATICALLY GENERATED
% INSTEAD, EDIT THE FILE HERE:
%
% https://docs.google.com/spreadsheets/d/1qRaEKxhyfwjHDWa3burGeqK0zeC00Br8NhJfnMWXJdk/edit?usp=sharing
%
% If you edit this file, your changes will be overwritten from this
% spreadsheet linked above.
%%%%%%%%%%%%%%%%%%%%%%%%%%%%%%%%%%%%%%%%%%%%%%%%%%%%%%%
\begin{enumerate}
\item \textbf{Efficient Systems for AI Applications (AI 216)}: This course introduces students to the computer system fundamentals that underpin AI systems.  Students learn how memory and parameters are organized in low-level systems and the programming techniques necessary to effeciently update those parameters on large datasets.  Students also learn how the mathematical fundamentals of AI algorithms are translated into learned parameters given large datasets.  Finally, students learn how modern algorithms update those parameters effeciently with large datasets that exceed the capacity of individual computers. [Linear algebra is required for this course but may be taken concurrently.]
\item \textbf{Measuring Preferences and Rankings (AI 220)}: A key component of modern AI systems is knowing when one AI system (or its output) is better than another.  Many of the mathematical tools used to make these decisions come from psychology.  This course introduces the mathematical tools that form the foundation of these techniques, teaches students to fit these models to data, and then explains how these models are used in modern AI systems and in AI evaluations.
\item \textbf{Multilingual Text Processing and Evaluation (AI 370)}: This course covers the representation and manipulation of linguistic data on computers.  After establishing the fundamentals of byte-level representation and how different languages are represented on a computer, the course discusses the type / token distinction of word use and how this is reflected in computational representations of language and how this is complicated by languages with implicit or ambiguous tokenization.  Finally, this course covers the application of large models (taken as a given) and resources to novel tasks or in novel combinations: how to adapt existing resources to new tasks, how to evaluate how well the models perform, and how to efficiently adapt models to these tasks.  
\item \textbf{Multimodal Generation (AI 429)}: Introduction to the algorithms that allow the generation of image, sound, video, and other modalities based on instructions, the theory behind joint representations that bridge the divide between description and representation and how they are rendered in the target medium, how to iteratively refine and render the target representations, and how to train and fine-tune the models to be more expressive or cover new concepts.  Discussion of dataset curation, evaluation, detection, and social impact of these algorithms.
\item \textbf{Capstone in Artificial Intelligence (AI 473)}: Students will be paired with project advisors from the UMD faculty or alternatively, an industry advisor. Students are encouraged to plan for projects results that can be published at academic conferences or will impact academic research.
Semester-long project course in which each student will identify and carry out a project related to machine learning, with the goal of publishing a research paper or software tool.
\end{enumerate}