
%%%%%%%%%%%%%%%%%%%%%%%%%%%%%%%%%%%%%%%%%%%%%%%%%%%%%%
% DO NOT EDIT THIS FILE, IT IS AUTOMATICALLY GENERATED
% INSTEAD, EDIT THE FILE HERE:
%
% https://docs.google.com/spreadsheets/d/1qRaEKxhyfwjHDWa3burGeqK0zeC00Br8NhJfnMWXJdk/edit?usp=sharing
%
% If you edit this file, your changes will be overwritten from this
% spreadsheet linked above.
%%%%%%%%%%%%%%%%%%%%%%%%%%%%%%%%%%%%%%%%%%%%%%%%%%%%%%%
\begin{enumerate}
\item \textbf{Efficient Systems for AI Applications (\prefix{} 216)}: This course introduces students to the computer system fundamentals that underpin AI systems.  Students learn how memory and parameters are organized in low-level systems and the programming techniques necessary to effeciently update those parameters on large datasets.  Students also learn how the mathematical fundamentals of AI algorithms are translated into learned parameters given large datasets.  Finally, students learn how modern algorithms update those parameters effeciently with large datasets that exceed the capacity of individual computers. [Linear algebra is required for this course but may be taken concurrently.] Prereqs: (ENEE 290 OR MATH 243 OR MATH 461 OR MATH 240 OR DATA 250 OR MATH 341) AND (MATH 340 OR MATH 140) AND (CMSC 132 OR CMSC 142)
\item \textbf{Measuring Preferences and Rankings (\prefix{} 220)}: A key component of modern AI systems is knowing when one AI system (or its output) is better than another.  Many of the mathematical tools used to make these decisions come from psychology.  This course introduces the mathematical tools that form the foundation of these techniques, teaches students to fit these models to data, and then explains how these models are used in modern AI systems and in AI evaluations. Prereqs: MATH 340 OR MATH 140
\item \textbf{Multilingual Text Processing and Evaluation (\prefix{} 370)}: This course covers the representation and manipulation of linguistic data on computers.  After establishing the fundamentals of byte-level representation and how different languages are represented on a computer, the course discusses the type / token distinction of word use and how this is reflected in computational representations of language and how this is complicated by languages with implicit or ambiguous tokenization.  Finally, this course covers the application of large models (taken as a given) and resources to novel tasks or in novel combinations: how to adapt existing resources to new tasks, how to evaluate how well the models perform, and how to efficiently adapt models to these tasks.   Prereqs: (\prefix{} 216) AND (\prefix{} 221 OR CMSC 421)
\item \textbf{Reinforcement Learning (\prefix{} 427)}: A survey of the theory and practice of reinforcement learning algorithms: how to formulate reward functions, how to estimate policies given rewards, and how to estimate the value of states given a reward survey.  After introducing the foundations of reinforcement learning, the course covers imitation learning, opponent modeling, and deep Q-learning. Prereqs: (\prefix{} 216) AND (\prefix{} 221 OR CMSC 421)
\item \textbf{Multimodal Generation (\prefix{} 429)}: Introduction to the algorithms that allow the generation of image, sound, video, and other modalities based on instructions, the theory behind joint representations that bridge the divide between description and representation and how they are rendered in the target medium, how to iteratively refine and render the target representations, and how to train and fine-tune the models to be more expressive or cover new concepts.  Discussion of dataset curation, evaluation, detection, and social impact of these algorithms. Prereqs: (\prefix{} 216) AND (\prefix{} 221 OR CMSC 421)
\item \textbf{AI and the Life of Great Cities  (\prefix{} 460)}: "AI, Data Science and Cities" explores the transformative role of artificial intelligence
and data science in shaping urban environments, from smart infrastructure to
data-driven policymaking. Students will learn how to develop and apply AI algorithms
and data science pipelines to enhance mobility, sustainability, and governance while
also considering ethical, social, and equity challenges. Through case studies, hands-on
projects, and interdisciplinary discussions, the course equips students with the technical
and analytical skills needed to design AI- and data-driven urban solutions. By the end,
students will critically assess the potential and limitations of AI and Data Science in
building more resilient, inclusive, and intelligent cities. Prereqs: (\prefix{} 216) AND (\prefix{} 221 OR CMSC 421)
\item \textbf{Multiagent Systems (\prefix{} 461)}: A course on the theory and practice of AI agents interacting in the wild: how to ensure security, how communication protocols can emerge, and how to measure consensus. Prereqs: (\prefix{} 216) AND (\prefix{} 221 OR CMSC 421)
\item \textbf{Capstone in Artificial Intelligence (\prefix{} 473)}: Students will be paired with project advisors from the UMD faculty or alternatively, an industry advisor. Students are encouraged to plan for projects results that can be published at academic conferences or will impact academic research.
Semester-long project course in which each student will identify and carry out a project related to machine learning, with the goal of publishing a research paper or software tool. Prereqs: (\prefix{} 216) AND (\prefix{} 221 OR CMSC 421)
\item \textbf{AI Clinic (\prefix{} 491)}: The AI clinic aims to become the go-to resource for assisting local Maryland
governments and small organizations in their procurement processes for AI tools. The
idea behind the AI Clinic is inspired by legal clinics present across many law schools in
the US as well as by the Clinic to End Tech Abuse (CETA) at Cornell Tech. The AI Clinic
will provide services to local government agencies and small organizations that do not
have the expertise or resources to carry out the evaluation of AI tools that they are
considering using to support their decision-making processes.
The AI Clinic will take on “AI cases” that will be led by one faculty member and by a
group of four students with diverse educational backgrounds. AI cases will provide
semester-long, hands-on learning opportunities for students and will prepare them for
the challenges of AI deployment in real-world settings. The output of the AI cases might
include reports with tool assessments (technical and community-centric) that will be
shared with the agency or organization that requested the case, toolkits for decision
makers on how to carry out their own AI evaluations, or training sessions for decision
makers and small organizations on how to use different types of AI tools, among others. Prereqs: (\prefix{} 216) AND (\prefix{} 221 OR CMSC 421)
\end{enumerate}