\begin{enumerate}
\item \textbf{Introduction to AI and the Law (AI 102)}: This course will examine AI Law and Regulation from the perspective of US Law and industry self-regulation with some discussion of select relevant international law and voluntary frameworks (i.e., EU AI Act and ISO/UN/OECD). Specific topics of US and international law may include: American Legal Process and Procedure, Administrative Law for AI Regulation, Constitutional Law Relationships to AI, Intellectual Property Law (Copyright, Trademark, Patent, and Trade Secret Law), Negligence/Products Liability Law, Privacy Law and Privacy Torts, Fundamentals of Contract Law, Industry Self-Regulation and International Frameworks.
\item \textbf{Introduction to AI and Food (AI 103)}: This course will examine the role of AI in American Food.  From how chemical synthesis driven by AI can produce new fertilizers, insecticides, and other tools that can improve the yield and productivity of crops, to using computer vision to monitor livestock, to harvesting crops with robotic agents, this course will explore the role of AI in food production.  Next, the course will cover the role of AI in food storage, transportation, and its trade on commodities markets.  Finally, the course will end with an overview of AI in food preparation, marketing, and helping individuals making nutritional decisions.
\item \textbf{Introduction to AI and Creativity (AI 104)}: This course will examine the role of AI in creative---particularly visual---pursuits.  Student will learn how to generate and edit images and videos using AI tools and learn the limitations of those tools.  Students will learn how to add additional information to these models and learn how to detect AI-created images and videos.  The final project will be to create a portfolio of artwork assisted by AI tools.
\item \textbf{Efficient Systems for AI Applications (AI 216)}: This course introduces students to the computer system fundamentals that underpin AI systems.  Students learn how memory and parameters are organized in low-level systems and the programming techniques necessary to effeciently update those parameters on large datasets.  Students also learn how the mathematical fundamentals of AI algorithms are translated into learned parameters given large datasets.  Finally, students learn how modern algorithms update those parameters effeciently with large datasets that exceed the capacity of individual computers.
\item \textbf{Measuring Preferences and Rankings (AI 220)}: A key component of modern AI systems is knowing when one AI system (or its output) is better than another.  Many of the mathematical tools used to make these decisions come from psychology.  This course introduces the mathematical tools that form the foundation of these techniques, teaches students to fit these models to data, and then explains how these models are used in modern AI systems and in AI evaluations.
\item \textbf{Classical AI Algorithms (AI 221)}: This course introduces the foundation of AI theory and practice.  Students learn how to search over structured representations to find optimal solutions to planning problems, how to represent real-world problems in graph structures or game trees, and how to represent and solve AI problems using first-order logic.
\item \textbf{nan (AI 322)}: This course presents the foundational algorithms of using data to learn algorithms and representations to solve real-world problems.  The course begins by covering representations of examples using learned distributed representations and pre-defined sparse representations, and then progresses to learning classifiers from those representations.  The course then moves to algorithms for learning these classifiers from larger datasets and how to debug whether the learned algorithms are learning effectively from the data.
\item \textbf{Multilingual Text Processing and Evaluation (AI 370)}: This course covers the representation and manipulation of linguistic data on computers.  After establishing the fundamentals of byte-level representation and how different languages are represented on a computer, the course discusses the type / token distinction of word use and how this is reflected in computational representations of language and how this is complicated by languages with implicit or ambiguous tokenization.  Finally, this course covers the application of large models (taken as a given) and resources to novel tasks or in novel combinations: how to adapt existing resources to new tasks, how to evaluate how well the models perform, and how to efficiently adapt models to these tasks.  
\item \textbf{Multimodal Generation (AI 429)}Description TBD
\end{enumerate}