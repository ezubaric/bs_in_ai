
%%%%%%%%%%%%%%%%%%%%%%%%%%%%%%%%%%%%%%%%%%%%%%%%%%%%%%
% DO NOT EDIT THIS FILE, IT IS AUTOMATICALLY GENERATED
% INSTEAD, EDIT THE FILE HERE:
%
% https://docs.google.com/spreadsheets/d/1qRaEKxhyfwjHDWa3burGeqK0zeC00Br8NhJfnMWXJdk/edit?usp=sharing
%
% If you edit this file, your changes will be overwritten from this
% spreadsheet linked above.
%%%%%%%%%%%%%%%%%%%%%%%%%%%%%%%%%%%%%%%%%%%%%%%%%%%%%%%
\begin{enumerate}
\item \textbf{Introduction to Data Science (CMSC 320)}: An introduction to the data science pipeline, i.e., the end-to-end process of going from unstructured, messy data to knowledge and actionable insights. Provides a broad overview of several topics including statistical data analysis, basic data mining and machine learning algorithms, large-scale data management, cloud computing, and information visualization. Prereqs: (Programming2: CMSC 131 OR INST 326 OR CMSC 141) AND (Linalg: MATH 141)
\item \textbf{Algorithms (CMSC 351)}: A systematic study of the complexity of some elementary algorithms related to sorting, graphs and trees, and combinatorics. Algorithms are analyzed using mathematical techniques to solve recurrences and summations. Prereqs: MATH 340 OR MATH 140 AND CMSC 131 OR INST 326 OR CMSC 141
\item \textbf{Introduction to Artificial Intelligence (CMSC 421)}: Introduces a range of ideas and methods in AI, varying semester to semester but chosen largely from: automated heuristic search, planning, games, knowledge representation, logical and statistical inference, learning, natural language processing, vision, robotics, cognitive modeling, and intelligent agents. Programming projects will help students obtain a hands-on feel for various topics.  Credit only granted to one of 421 or 221. Prereqs: Prerequisite: Minimum grade of C- in CMSC351 and CMSC330; and permission of CMNS-Computer Science department. 
\item \textbf{Computer Vision (CMSC 426)}: An introduction to basic concepts and techniques in computervision. This includes low-level operations such as image filtering and edge detection, 3D reconstruction of scenes using stereo and structure from motion, and object detection, recognition and classification. Prereqs: (Compgraphs: MATH 461 OR MATH 241 OR MATH 341 OR DATA 250 OR MATH 240 AND MATH 340 OR MATH 140 AND CMSC 142 OR CMSC 132) AND (Gofai: CMSC 131 OR INST 326 OR CMSC 141)
\item \textbf{Computer Graphics (CMSC 427)}: An introduction to 3D computer graphics, focusing on the underlying building blocks and algorithms for applications such as 3D computer games, and augmented and virtual reality (AR/VR). Covers the basics of 3D image generation and 3D modeling, with an emphasis on interactive applications. Discusses the representation of 3D geometry, 3D transformations, projections, rasterization, basics of color spaces, texturing and lighting models, as well as programming of modern Graphics Processing Units (GPUs). Includes programming projects where students build their own 3D rendering engine step-by-step. Prereqs: (Compgraphs: MATH 461 OR MATH 241 OR MATH 341 OR DATA 250 OR MATH 240 AND MATH 340 OR MATH 140 AND CMSC 142 OR CMSC 132) AND (Gofai: CMSC 131 OR INST 326 OR CMSC 141)
\item \textbf{Natural Language Processing (CMSC 470)}: Introduction to fundamental techniques for automatically processing and generating natural language with computers. Machine learning techniques, models, and algorithms that enable computers to deal with the ambiguity and implicit structure of natural language. Application of these techniques in a series of assignments designed to address a core application such as question answering or machine translation.
 Prereqs: (Compgraphs: MATH 461 OR MATH 241 OR MATH 341 OR DATA 250 OR MATH 240 AND MATH 340 OR MATH 140 AND CMSC 142 OR CMSC 132) AND (Gofai: CMSC 131 OR INST 326 OR CMSC 141)
\item \textbf{Introduction to Deep Learning (CMSC 472)}: This course is an elementary introduction to a machine learning technique called deep learning, as well as its applications to a variety of domains. Along the way, the course also provides an intuitive introduction to machine learning such as simple models, learning paradigms, optimization, overfitting, importance of data, training caveats, etc. The assignments explore key concepts and simple applications, and the final project allows an in-depth exploration of a particular application area. By the end of the course, you will have an overview on the deep learning landscape and its applications. You will also have a working knowledge of several types of neural networks, be able to implement and train them, and have a basic understanding of their inner workings. Credit only granted to one of 472 or 372. Prereqs: (CMSC320) AND ( CMSC330) AND ( and CMSC351) AND ( MATH240)
\end{enumerate}