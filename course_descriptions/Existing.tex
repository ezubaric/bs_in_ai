
%%%%%%%%%%%%%%%%%%%%%%%%%%%%%%%%%%%%%%%%%%%%%%%%%%%%%%
% DO NOT EDIT THIS FILE, IT IS AUTOMATICALLY GENERATED
% INSTEAD, EDIT THE FILE HERE:
%
% https://docs.google.com/spreadsheets/d/1qRaEKxhyfwjHDWa3burGeqK0zeC00Br8NhJfnMWXJdk/edit?usp=sharing
%
% If you edit this file, your changes will be overwritten from this
% spreadsheet linked above.
%%%%%%%%%%%%%%%%%%%%%%%%%%%%%%%%%%%%%%%%%%%%%%%%%%%%%%%
\begin{enumerate}
\item \textbf{(Dis)ability in American Film (AMST 320)}: Explores the connection between film and disability through an analysis of independent and mainstream American films in various film genres. Specifically, we will consider how these film representations reflect and/or challenge the shifting social perspectives of disability over the 20th and 21st centuries. Beginning with the presentation of disability as theatrical spectacle in the traveling sideshow and early cinema, we will work our way through film history to develop an understanding of our society's complicated relationship with disability.
 Prereqs: (Compgraphs: BSAI 216) AND (Gofai: CMSC 421 OR BSAI 221)
\item \textbf{Object-Oriented Programming I (CMSC 131)}: Introduction to programming and computer science. Emphasizes understanding and implementation of applications using object-oriented techniques. Develops skills such as program design and testing as well as implementation of programs using a graphical IDE. Programming done in Java.
\item \textbf{Object-Oriented Programming II (CMSC 132)}: Introduction to use of computers to solve problems using software engineering principles. Design, build, test, and debug medium -size software systems and learn to use relevant tools. Use object-oriented methods to create effective and efficient problem solutions. Use and implement application programming interfaces (APIs). Programming done in Java. Prereqs: CMSC 131 OR CMSC 141 OR INST 326
\item \textbf{Discrete Structures (CMSC 250)}: Fundamental mathematical concepts related to computer science, including finite and infinite sets, relations, functions, and propositional logic. Introduction to other techniques, modeling and solving problems in computer science. Introduction to permutations, combinations, graphs, and trees with selected applications. Prereqs: (Calc1: MATH 140 OR MATH 340) AND (Programming1: CMSC 131 OR CMSC 141 OR INST 326)
\item \textbf{Algorithms for Geospatial Computing (CMSC 401)} Prereqs: (Compgraphs: BSAI 216) AND (Gofai: CMSC 421 OR BSAI 221)
\item \textbf{Intro to Machine Learning (CMSC 422)}: TEMP Prereqs: (Compgraphs: BSAI 216) AND (Gofai: CMSC 421 OR BSAI 221)
\item \textbf{Game Programming (CMSC 425)} Prereqs: (Compgraphs: BSAI 216) AND (Gofai: CMSC 421 OR BSAI 221)
\item \textbf{Computer Vision (CMSC 426)} Prereqs: (Compgraphs: BSAI 216) AND (Gofai: CMSC 421 OR BSAI 221)
\item \textbf{Introduction to Human-Computer Interaction (CMSC 434)}: Assess usability by quantitative and qualitative methods. Conduct task analyses, usability tests, expert reviews, and continuing assessments of working products by interviews, surveys, and logging. Apply design processes and guidelines to develop professional quality user interfaces. Build low-fidelity paper mockups, and a high-fidelity prototype using contemporary tools such as graphic editors and a graphical programming environment (eg: Visual Basic, Java).
 Prereqs: (Compgraphs: BSAI 216) AND (Gofai: CMSC 421 OR BSAI 221)
\item \textbf{Programming Handheld Systems (CMSC 436)} Prereqs: (Compgraphs: BSAI 216) AND (Gofai: CMSC 421 OR BSAI 221)
\item \textbf{Algorithms for Data Science (CMSC 454)}: Fundamental methods for processing a high volume of data. Methods include stream processing, locally sensitive hashing, web search methods, page rank computation, network and link analysis, dynamic graph algorithms as well as methods to handle high dimensional data/dimensionality reduction.
 Prereqs: (Compgraphs: BSAI 216) AND (Gofai: CMSC 421 OR BSAI 221)
\item \textbf{Introduction to Natural Language Processing (CMSC 470)}: Introduction to fundamental techniques for automatically processing and generating natural language with computers. Machine learning techniques, models, and algorithms that enable computers to deal with the ambiguity and implicit structure of natural language. Application of these techniques in a series of assignments designed to address a core application such as question answering or machine translation.
 Prereqs: (Compgraphs: BSAI 216) AND (Gofai: CMSC 421 OR BSAI 221)
\item \textbf{Intro to Data Visualization (CMSC 471)} Prereqs: (Compgraphs: BSAI 216) AND (Gofai: CMSC 421 OR BSAI 221)
\item \textbf{Introduction to Deep Learning (CMSC 472)}: TEMP Prereqs: (Compgraphs: BSAI 216) AND (Gofai: CMSC 421 OR BSAI 221)
\item \textbf{Introduction to Computational Game Theory (CMSC 474)} Prereqs: (Compgraphs: BSAI 216) AND (Gofai: CMSC 421 OR BSAI 221)
\item \textbf{Robotics Perception and Planning (CMSC 477)}: A hands-on introduction to perception and planning for robotics, including rigid body transformations and rotations, dynamics and control of mobile robots/drones, graph based and sampling based planning algorithms, Bayseian and Kalman filtering, camera models and calibration, projective geometry, visual features, optical flow, pose estimation, RANSAC and Hough transform, structure from motion, visual odometry, machine learning basics, visual recognition and learning. Prereqs: (Compgraphs: BSAI 216) AND (Gofai: CMSC 421 OR BSAI 221)
\item \textbf{Selected Topics in Computer Science; Robotics (CMSC 498E)}: Overview on fundamental components of robotic systems, including the sensing and actuation, control and modeling of motion and perception, dynamics and kinematics, motion planning and manipulation of robots.
 Prereqs: (Compgraphs: BSAI 216) AND (Gofai: CMSC 421 OR BSAI 221)
\item \textbf{Selected Topics in Computer Science; Statistical Inference and Machine Learning Methods for Genomics Data (CMSC 498Y)}: Covers statistical inference and machine learning methods for analyzing genomic data. Examples of topics covered will include maximum likelihood(including composite and pseudo-likelihood functions), expectation-maximization, clustering algorithms, hidden markov models, statistical testing, MCMC and variational inference. Our focus will be on how these techniques are utilized to solve biological problems and the practical challenges that arise when analyzing large genomic data sets.
 Prereqs: (Compgraphs: BSAI 216) AND (Gofai: CMSC 421 OR BSAI 221)
\item \textbf{Selected Topics in Computer Science; Robotics (CMSC498E)}: TEMP Prereqs: (Compgraphs: BSAI 216) AND (Gofai: CMSC 421 OR BSAI 221)
\item \textbf{Discrete Mathematics (DATA 250)}: The art and science of discrete mathematics (not Calculus) and rigorous arguments related to
programming and data. Logic, set theory, and formal proofs will be covered in the first portion of the
course, and the second portion will be basic linear algebra, including matrices and systems of equations.
Emphasis on examples and applications so there will be some programming and other uses of the
computer. Classtime consists of lectures, group work, and other classwork. Prereqs: (Calc1: MATH 140 OR MATH 340) AND (Programming1: CMSC 131 OR CMSC 141 OR INST 326)
\item \textbf{Introduction to Data Science (DATA 320)}: An introduction to data science i.e., the end-to-end process of going from unstructured, messy data to knowledge and actionable insights. Provides a broad overview of several topics including statistical data analysis, basic data mining and machine learning algorithms, large-scale data management, cloud computing, and information visualization. Prereqs: (Programming2: CMSC 142 OR CMSC 132) AND (Linalg: MATH 243 OR MATH 461 OR ENEE 290 OR MATH 240 OR DATA 250 OR MATH 341)
\item \textbf{Robotics Programming (ENAE 450)}: This practical robotics class teaches practical skills to build, control, and deploy robotic systems. Interdisciplinary groups of students develop real-world robotic systems. The first 10 weeks of the lab are devoted to 5 pre-programmed experiments. The remainder of the lab is devoted to student projects. Students work in teams of 2, preferably with each student coming from a different background in engineering. There are 2 weekly lectures. The emphasis of the class is entirely on making a real robot do what you want it to do. In the first experiment, students perform a simple servomechanism experiment, where students control a single joint of a robot. We vary the weight on the movable rod to simulate the effects of the changing inertia due to outer segments moving. Next, we have the students directly control several joints of a robot arm. The third experiment is to control the position and orientation of the end effector. A fourth experiment deals with grasping. A fifth experiment deals with the position and orientation of a wheeled robot.  Prereqs: (Compgraphs: BSAI 216) AND (Gofai: CMSC 421 OR BSAI 221)
\item \textbf{Introduction to Differential Equations and Linear Algebra for Engineers (ENEE 290)}: First-order differential equations, matrices and systems of linear equations, finite-dimensional vector spaces, inner product spaces, eigenvalues and eigenvectors, linear differential equations of higher order, and systems of differential equations. This course covers important topics in mathematics for Electrical and Computer Engineers. Specifically, several topics are covered, including first-order differential equations, matrices and systems of linear equations, finite-dimensional vector spaces, inner product spaces, eigenvalues and eigenvectors, linear differential equations of higher order, and systems of differential equations. Theoretical topics presented in the lectures will be reinforced by laboratory exercises.
 Prereqs: MATH 141
\item \textbf{Robotics Project Laboratory (ENEE 467)}: This practical robotics class teaches practical skills to build, control, and deploy robotic systems. Interdisciplinary groups of students develop real-world robotic systems. The first 10 weeks of the lab are devoted to 5 pre-programmed experiments. The remainder of the lab is devoted to student projects. Students work in teams of 2, preferably with each student coming from a different background in engineering. There are 2 weekly lectures. The emphasis of the class is entirely on making a real robot do what you want it to do. In the first experiment, students perform a simple servomechanism experiment, where students control a single joint of a robot. We vary the weight on the movable rod to simulate the effects of the changing inertia due to outer segments moving. Next, we have the students directly control several joints of a robot arm. The third experiment is to control the position and orientation of the end effector. A fourth experiment deals with grasping. A fifth experiment deals with the position and orientation of a wheeled robot.  Prereqs: (Compgraphs: BSAI 216) AND (Gofai: CMSC 421 OR BSAI 221)
\item \textbf{Assistive Robotics (ENME  444)} Prereqs: (Compgraphs: BSAI 216) AND (Gofai: CMSC 421 OR BSAI 221)
\item \textbf{Introduction to Robotics (ENME 480)}: This course is an introductory course to the robotics minor and educates students in the elementary concepts of robotics. The topics covered in the course include mathematics of rigid motion, rotations, translations, homogeneous transformations, forward kinematics, inverse kinematics, velocity kinematics, geometric Jacobian, analytical Jacobian, motion planning, trajectory generation, independent joint control, linear control methods such as PD, PID, actuator dynamics, feedforward control for trajectory tracking, force control, basic computer vision concepts including thresholding, image segmentation, and camera calibration. This course also includes a laboratory component to be conducted in the Robot Realization Laboratory in the Engineering Annex Building. Prereqs: (Compgraphs: BSAI 216) AND (Gofai: CMSC 421 OR BSAI 221)
\item \textbf{Special Topics in Geography; Introduction to Spatial Artificial Intelligence (GEOG 398E)}: An introductory course to spatial artificial intelligence (AI), providing a big picture of spatial AI applications (e.g., Google Maps, Uber/Lyft, Earth observation, smart cities, autonomous vehicles), techniques, platforms, trends, debates, etc. The course will cover basics of AI, identify challenges faced by AI techniques in the context of spatial data and applications, and introduce spatial-aware AI methods to address them. AI topics include but are not limited to: spatial data models and knowledge representation, pattern mining, machine learning, perception, planning, etc. Students are expected to have a broad understanding of spatial A concepts, develop intuitions and insights to AI techniques, and have some hands-on experience (Python) at the end of the course. Prereqs: (Compgraphs: BSAI 216) AND (Gofai: CMSC 421 OR BSAI 221)
\item \textbf{Special Topics in Immersive Media; Creative Experiments with AI (IMDM 498E)} Prereqs: (Compgraphs: BSAI 216) AND (Gofai: CMSC 421 OR BSAI 221)
\item \textbf{Designing Fair Systems (INST 204)}: Reviews how specific values are built into different automated decision-making systems as an inevitable result of constructing mechanisms meant to produce specific outcomes. These values create differential outcomes for the different people enmeshed in these systems, but both these values and these systems can be changed to support different values and different outcomes. The class serves as an introduction to the emerging field of algorithmic bias that bridges the disciplines of information science, computer science, law, policy, philosophy, sociology, urban planning, and others. Prereqs: (Introsem: BSAI 101) AND (Programming1: CMSC 131 OR CMSC 141 OR INST 326)
\item \textbf{Object-Oriented Programming for Information Science (INST 326)}: An introduction to programming, emphasizing understanding and
implementation of applications using object-oriented techniques. Topics to
be covered include program design and testing as well as implementation of
programs.
\item \textbf{User-Centered Design (INST 362)}: Introduction to human-computer interaction (HCI), with a focus on how HCI connects psychology, information systems, computer science, and human factors. User-centered design and user interface implementation methods discussed include identifying user needs, understanding user behaviors, envisioning interfaces, and utilizing prototyping tools, with an emphasis on incorporating people in the design process from initial field observations to summative usability testing.

\item \textbf{Designing Patient-Centered Technologies (INST 402)}: Companies have created a vast array of apps and other technologies for understanding managing personal health and wellness, but many of them have been created with little understanding of audience needs or potential ethical issues. Course introduces students to the unique challenges of studying people's health and wellness needs as well as designing and evaluating technologies to meet those needs.
\item \textbf{Data Science Techniques (INST 414)}: An exploration of how to extract insights from large-scale datasets. The course will cover the complete analytical funnel from data extraction and cleaning to data analysis and insights interpretation and visualization. The data analysis component will focus on techniques in both supervised and unsupervised learning to extract information from datasets. Topics will include clustering, classification, and regression techniques. Through homework assignments, a project, exams and in-class activities, students will practice working with these techniques and tools to extract relevant information from structured and unstructured data. Prereqs: (Programming2: CMSC 142 OR CMSC 132) AND (Linalg: MATH 243 OR MATH 461 OR ENEE 290 OR MATH 240 OR DATA 250 OR MATH 341)
\item \textbf{Emerging Technologies and Risk Management (INST 461)} Prereqs: (Compgraphs: BSAI 216) AND (Gofai: CMSC 421 OR BSAI 221)
\item \textbf{Introductory Linguistics (LING 200)}: An exploration of the nature of human language. Introduction to the basic concepts and methodology of modern linguistic analysis (sound systems, word formation, sentence structure). Examination of the factors that contribute to dialect differences and the social implications of language variation. Additional topics may include: semantics, pragmatics, language change, writing systems, typology, language universals, comparison with other communication systems. Prereqs: (Compgraphs: BSAI 216) AND (Gofai: CMSC 421 OR BSAI 221)
\item \textbf{Language and Mind (LING 240)}: The study of language as a cognitive phenomenon. Ways of representing people's knowledge of their native language, ways in which that knowledge is attained naturally by children, and how it is used in speaking and listening. Additional topics may include: animal communication, language and the brain, language and thought. Prereqs: (Compgraphs: BSAI 216) AND (Gofai: CMSC 421 OR BSAI 221)
\item \textbf{Syntax I (LING 311)}: Basic concepts, analytical techniques of generative syntax, relation to empirical limits imposed by viewing grammars as representations of a component of human mind. Aspects of current theories. Prereqs: (Compgraphs: BSAI 216) AND (Gofai: CMSC 421 OR BSAI 221)
\item \textbf{Phonetics (LING 320)}: Representations and models of acoustic and articulatory phonetics. Develops concepts and skills for description, measurement and scientific analysis of the sound systems of human languages, including various varieties of English.

\item \textbf{Phonology I (LING 321)}: Properties of sound systems of human languages, basic concepts and analytical techniques of generative phonology. Empirical limits imposed by viewing grammars as cognitive representations. Physiological properties and phonological systems; articulatory phonetics and distinctive feature theory.
\item \textbf{Phonology II (LING 322)}: Continuation of LING321. Further investigation of phonological phenomena and phonological theory. Revising and elaborating the theory of the phonological representation; interaction of phonology and morphology.
\item \textbf{Grammar and Meaning (LING 410)}: The basic notions of semantic theory: reference, quantification, scope relations, compositionality, thematic relations, tense and time, etc. The role these notions play in grammars of natural languages. Properties of logical form and relationship with syntax. Prereqs: (Compgraphs: BSAI 216) AND (Gofai: CMSC 421 OR BSAI 221)
\item \textbf{Child Language Acquisition (LING 444)}: Examines language acquisition in infancy and early childhood: the nature of children's linguistic representations and how these develop naturally. Role of (possible) innate linguistic structure and interaction of such structure with experience. Evaluation of methods and results of current and classic research leading to contemporary models of language development.
\item \textbf{Calculus I (MATH 140)}: Introduction to calculus, including functions, limits, continuity, derivatives and applications of the derivative, sketching of graphs of functions, introduction to definite and indefinite integrals, and calculation of area. The course is especially recommended for science and mathematics majors. Credit will be granted for only one of the following: MATH 140 or MATH 136 or MATH 120.
\item \textbf{Calculus II (MATH 141)}: Continuation of MATH140, including techniques of integration, improper integrals, applications of integration (such as volumes, work, arc length, moments), inverse functions, exponential and logarithmic functions, sequences and series.
\item \textbf{Introduction to Linear Algebra (MATH 240)}: Basic concepts of linear algebra: vector spaces, applications to line and plane geometry, linear equations and matrices, similar matrices, linear transformations, eigenvalues, determinants and quadratic forms. All sections of the course will use the software system MATLAB. Credit will be granted for only one of the following: MATH 240 or MATH 341 or MATH 461.
\item \textbf{Introduction to Linear Algebra and Differential Equations (MATH 243)}: The basics of linear algebra and differential equations, with an emphasis on general physical and
engineering applications. Aimed at students who need the material for future coursework but do not
need as much depth and rigor as provided by MATH240/MATH461 and MATH246.
\item \textbf{Multivariable Calculus, Linear Algebra and Differential Equations I (Honors) (MATH 340)}: This is the first semester of the two-semester honors sequence Math 340-341 which gives a unified and enriched treatment of multivariable calculus, linear algebra, and ordinary differential equations, with supplementary material from differential geometry, Fourier series and calculus of variations.
\item \textbf{Multivariable Calculus, Linear Algebra and Differential Equations II (MATH 341)}: This course is a continuation of MATH 340. The honors sequence MATH 340-341 covers roughly the same material of MATH 240, 241, and 246, but in greater depth and rigor. This semeseter will begin with remaining material from Multivariable Calculus (MATH 241) on extrema of functions of several variables and Lagrange multipliers. The remainder of the semester, and the bulk of the course, is then devoted to the theory of Ordinary Differential Equations (MATH 246). Prereqs: MATH 141
\item \textbf{Linear Algebra for Scientists and Engineers (MATH 461)}: The course provides an introduction to linear algebra and matrix theory. It is intended primarily for engineering students. This course cannot be used toward the upper level math requirements for MATH/STAT majors. Credit will be granted for only one of the following: MATH 240, MATH 341, or MATH 461.
\item \textbf{AI and Ethics (PHIL 211)}: An introduction to a major subfield of contemporary Philosophy, namely applied ethics, and the experience of using some major tools in the practice of philosophy more generally, namely, the construction and formal evaluation of arguments, conceptual analysis, the use of thought experiments, and clear, direct and persuasive writing. Learning how to execute the latter will involve an intense iterative process.
The substantive focus of the course will be the ethical evaluation of AI in some of its current and potentially future incarnations. We’ll examine algorithmic opacity, algorithmic bias
and decision-making, autonomous weapons systems, human-robot interaction, and artificial
moral agents, in order to uncover what, if any, ethical issues they give rise to. Prereqs: BSAI 101
\item \textbf{Public Leaders and Active Citizens (PLCY 201)} Prereqs: (Compgraphs: BSAI 216) AND (Gofai: CMSC 421 OR BSAI 221)
\item \textbf{Innovation and Social Change: Creating Change for Good (PLCY 215)} Prereqs: (Compgraphs: BSAI 216) AND (Gofai: CMSC 421 OR BSAI 221)
\item \textbf{Ethical, Policy and Social Implications for Science and Technology (PLCY 240)} Prereqs: (Compgraphs: BSAI 216) AND (Gofai: CMSC 421 OR BSAI 221)
\item \textbf{Human and Animal Intelligence (PSYC 431)} Prereqs: (Compgraphs: BSAI 216) AND (Gofai: CMSC 421 OR BSAI 221)
\item \textbf{Introduction to Sociology (SOCY 100)}: Introduces fundamental concepts and theories of sociology. Guided by C. Wright Mills' "sociological imagination," the course promotes critical thinking; challenges conventional assumptions about culture politics, history, and psychology; and equips students with theoretical approaches and research methods to analyze various sociological topics, including family, work, education, religion, social movements, and issues related to class, gender, race, and ethnic inequalities. Prereqs: (Compgraphs: BSAI 216) AND (Gofai: CMSC 421 OR BSAI 221)
\item \textbf{Social Aspects of Artificial Intelligence (SOCY 216)}: In two generations computers insinuated themselves into the way societies create wealth, wage war, work, and govern their citizens. While scientists across disciplines debate the feasibility of engineering artificial general intelligence, the race is on to create computers (classic and quantum) that match or surpass human intelligence in as many domains as possible. Students in this course will weigh some social consequences of living with smart machines that are everywhere and never sleep, and confront the question of whether AI has gone too far, or not far enough. Prereqs: (Compgraphs: BSAI 216) AND (Gofai: CMSC 421 OR BSAI 221)
\item \textbf{Social Dimensions of Privacy and Surveillance (SOCY 455)} Prereqs: (Compgraphs: BSAI 216) AND (Gofai: CMSC 421 OR BSAI 221)
\item \textbf{Smart Machines and Human Prospects (SOCY 456)} Prereqs: (Compgraphs: BSAI 216) AND (Gofai: CMSC 421 OR BSAI 221)
\item \textbf{Digital Technology and Society (SOCY 462)} Prereqs: (Compgraphs: BSAI 216) AND (Gofai: CMSC 421 OR BSAI 221)
\item \textbf{Applied Probability and Statistics I (STAT 400)}: Stat 400 is an introductory course to probability, the mathematical theory of randomness, and to statistics, the mathematical science of data analysis and analysis in the presence of uncertainty. Applications of statistics and probability to real world problems are also presented. Prereqs: MATH 140 OR MATH 340
\item \textbf{Machine Learning (STAT 426)} Prereqs: (Programming2: CMSC 142 OR CMSC 132) AND (Linalg: MATH 243 OR MATH 461 OR ENEE 290 OR MATH 240 OR DATA 250 OR MATH 341)
\item \textbf{Introduction to Disability Studies (WGSS 105)}: This course will introduce students to theories of disability justice as they intersect with feminist and antiracist struggles. Tracing the emergence of the concept of disability alongside the rise of racial knowledge since the 19th century, we will consider how disability activists have responded to ableism by developing art, political strategies, and subcultures that promote a more just society built for a wider variety of human bodies. Students will learn about the moral, medical, social, and ecological models of disability; explore varied disability experiences relating to mental illness, chronic disease, and sensory and mobility impairments; debate ethical questions concerning eugenics, selective abortion, health care access, and medical technologies; and analyze the work of disabled artists and activists of color. Students will also discuss principles of universal design which seek to make classrooms more just and collaborative. In order to balance accessibility and community building, the course has been designed for synchronous online instruction complemented by optional in-person sessions.
 Prereqs: (Compgraphs: BSAI 216) AND (Gofai: CMSC 421 OR BSAI 221)
\item \textbf{Gender, Race, and Computing (WGSS 115)} Prereqs: (Compgraphs: BSAI 216) AND (Gofai: CMSC 421 OR BSAI 221)
\end{enumerate}