
%%%%%%%%%%%%%%%%%%%%%%%%%%%%%%%%%%%%%%%%%%%%%%%%%%%%%%
% DO NOT EDIT THIS FILE, IT IS AUTOMATICALLY GENERATED
% INSTEAD, EDIT THE FILE HERE:
%
% https://docs.google.com/spreadsheets/d/1qRaEKxhyfwjHDWa3burGeqK0zeC00Br8NhJfnMWXJdk/edit?usp=sharing
%
% If you edit this file, your changes will be overwritten from this
% spreadsheet linked above.
%%%%%%%%%%%%%%%%%%%%%%%%%%%%%%%%%%%%%%%%%%%%%%%%%%%%%%%
\begin{enumerate}
\item \textbf{Multimodal Generation (AI 429)}: TEMP Prereqs: (DATA 250 OR MATH 241 OR MATH 341 OR MATH 240 OR MATH 461 AND MATH 340 OR MATH 140 AND CMSC 142 OR CMSC 132) AND (CMSC 141 OR CMSC 131 OR INST 326)
\item \textbf{Biomolecular Measurement and Data Analysis (BCHM 477)} Prereqs: (DATA 250 OR MATH 241 OR MATH 341 OR MATH 240 OR MATH 461 AND MATH 340 OR MATH 140 AND CMSC 142 OR CMSC 132) AND (CMSC 141 OR CMSC 131 OR INST 326)
\item \textbf{Introduction of Machine Learning in Chemical Engineering (CHBE 452)} Prereqs: (DATA 250 OR MATH 241 OR MATH 341 OR MATH 240 OR MATH 461 AND MATH 340 OR MATH 140 AND CMSC 142 OR CMSC 132) AND (CMSC 141 OR CMSC 131 OR INST 326)
\item \textbf{Object-Oriented Programming I (CMSC 131)}: Introduction to programming and computer science. Emphasizes understanding and implementation of applications using object-oriented techniques. Develops skills such as program design and testing as well as implementation of programs using a graphical IDE. Programming done in Java.
\item \textbf{Object-Oriented Programming II (CMSC 132)}: Introduction to use of computers to solve problems using software engineering principles. Design, build, test, and debug medium -size software systems and learn to use relevant tools. Use object-oriented methods to create effective and efficient problem solutions. Use and implement application programming interfaces (APIs). Programming done in Java. Prereqs: (MATH140) AND (CMSC131)
\item \textbf{Discrete Structures (CMSC 250)}: Fundamental mathematical concepts related to computer science, including finite and infinite sets, relations, functions, and propositional logic. Introduction to other techniques, modeling and solving problems in computer science. Introduction to permutations, combinations, graphs, and trees with selected applications. Prereqs: () AND ()
\item \textbf{Algorithms for Geospatial Computing (CMSC 401)}: TEMP Prereqs: (DATA 250 OR MATH 241 OR MATH 341 OR MATH 240 OR MATH 461 AND MATH 340 OR MATH 140 AND CMSC 142 OR CMSC 132) AND (CMSC 141 OR CMSC 131 OR INST 326)
\item \textbf{Intro to Machine Learning (CMSC 422)}: TEMP Prereqs: (DATA 250 OR MATH 241 OR MATH 341 OR MATH 240 OR MATH 461 AND MATH 340 OR MATH 140 AND CMSC 142 OR CMSC 132) AND (CMSC 141 OR CMSC 131 OR INST 326)
\item \textbf{Image Processing (CMSC 426)}: TEMP Prereqs: (DATA 250 OR MATH 241 OR MATH 341 OR MATH 240 OR MATH 461 AND MATH 340 OR MATH 140 AND CMSC 142 OR CMSC 132) AND (CMSC 141 OR CMSC 131 OR INST 326)
\item \textbf{Algorithms for Data Science (CMSC 454)}: TEMP Prereqs: (DATA 250 OR MATH 241 OR MATH 341 OR MATH 240 OR MATH 461 AND MATH 340 OR MATH 140 AND CMSC 142 OR CMSC 132) AND (CMSC 141 OR CMSC 131 OR INST 326)
\item \textbf{Natural Language Processing (CMSC 470)}: TEMP Prereqs: (DATA 250 OR MATH 241 OR MATH 341 OR MATH 240 OR MATH 461 AND MATH 340 OR MATH 140 AND CMSC 142 OR CMSC 132) AND (CMSC 141 OR CMSC 131 OR INST 326)
\item \textbf{Introduction to Deep Learning (CMSC 472)}: TEMP Prereqs: (DATA 250 OR MATH 241 OR MATH 341 OR MATH 240 OR MATH 461 AND MATH 340 OR MATH 140 AND CMSC 142 OR CMSC 132) AND (CMSC 141 OR CMSC 131 OR INST 326)
\item \textbf{Robotics Perception and Planning (CMSC 477)}: This course teaches the fundamentals of robot perception and robot path planning. The syllabus and course is divided into two segments, as per the major aspects involved in robotics. There will be lectures on (a) planning and control and (b) perception, with projects and homework. The syllabus includes the following: Motion Planning Introduction, Rigid Body Transformations, Velocity, Velocity Dynamics, Vehicle Controls, Graph Based Planning, Sampling Based Planning, Trajectory Planning, Navigation, Baeyesian and Kalman Filtering, Camera Model and Calibration, Projective Geometry, Visual Perception features, Optical Flow, Pose Estimation, 3D Velocities, Basics of Machine Learning, Structure from Motion, Visual Odometry, and Recognition and Learning. There are two examinations, 3 projects in multiple phases, and two homeworks. The class uses robots, mobile platforms with sensors and effectors. Drone experiments are done in simulation or in the Iribe Drone Lab, space permitting. Prereqs: (DATA 250 OR MATH 241 OR MATH 341 OR MATH 240 OR MATH 461 AND MATH 340 OR MATH 140 AND CMSC 142 OR CMSC 132) AND (CMSC 141 OR CMSC 131 OR INST 326)
\item \textbf{Algorithms for Geospatial Computing (CMSC401)} Prereqs: (DATA 250 OR MATH 241 OR MATH 341 OR MATH 240 OR MATH 461 AND MATH 340 OR MATH 140 AND CMSC 142 OR CMSC 132) AND (CMSC 141 OR CMSC 131 OR INST 326)
\item \textbf{Introduction to Machine Learning (CMSC422)} Prereqs: (DATA 250 OR MATH 241 OR MATH 341 OR MATH 240 OR MATH 461 AND MATH 340 OR MATH 140 AND CMSC 142 OR CMSC 132) AND (CMSC 141 OR CMSC 131 OR INST 326)
\item \textbf{Game Programming (CMSC425)} Prereqs: (DATA 250 OR MATH 241 OR MATH 341 OR MATH 240 OR MATH 461 AND MATH 340 OR MATH 140 AND CMSC 142 OR CMSC 132) AND (CMSC 141 OR CMSC 131 OR INST 326)
\item \textbf{Computer Vision (CMSC426)} Prereqs: (DATA 250 OR MATH 241 OR MATH 341 OR MATH 240 OR MATH 461 AND MATH 340 OR MATH 140 AND CMSC 142 OR CMSC 132) AND (CMSC 141 OR CMSC 131 OR INST 326)
\item \textbf{Algorithms for Data Science (CMSC454)} Prereqs: (DATA 250 OR MATH 241 OR MATH 341 OR MATH 240 OR MATH 461 AND MATH 340 OR MATH 140 AND CMSC 142 OR CMSC 132) AND (CMSC 141 OR CMSC 131 OR INST 326)
\item \textbf{Introduction to Natural Language Processing (CMSC470)} Prereqs: (DATA 250 OR MATH 241 OR MATH 341 OR MATH 240 OR MATH 461 AND MATH 340 OR MATH 140 AND CMSC 142 OR CMSC 132) AND (CMSC 141 OR CMSC 131 OR INST 326)
\item \textbf{Robotics Perception and Planning (CMSC477)} Prereqs: (DATA 250 OR MATH 241 OR MATH 341 OR MATH 240 OR MATH 461 AND MATH 340 OR MATH 140 AND CMSC 142 OR CMSC 132) AND (CMSC 141 OR CMSC 131 OR INST 326)
\item \textbf{Selected Topics in Computer Science; Robotics (CMSC498E)} Prereqs: (DATA 250 OR MATH 241 OR MATH 341 OR MATH 240 OR MATH 461 AND MATH 340 OR MATH 140 AND CMSC 142 OR CMSC 132) AND (CMSC 141 OR CMSC 131 OR INST 326)
\item \textbf{Selected Topics in Computer Science; Statistical Inference and Machine Learning Methods for Genomics Data (CMSC498Y)} Prereqs: (DATA 250 OR MATH 241 OR MATH 341 OR MATH 240 OR MATH 461 AND MATH 340 OR MATH 140 AND CMSC 142 OR CMSC 132) AND (CMSC 141 OR CMSC 131 OR INST 326)
\item \textbf{Discrete Mathematics (DATA 250)}: The art and science of discrete mathematics (not Calculus) and rigorous arguments related to
programming and data. Logic, set theory, and formal proofs will be covered in the first portion of the
course, and the second portion will be basic linear algebra, including matrices and systems of equations.
Emphasis on examples and applications so there will be some programming and other uses of the
computer. Classtime consists of lectures, group work, and other classwork. Prereqs: () AND ()
\item \textbf{Introduction to Data Science (DATA 320)}: An introduction to data science i.e., the end-to-end process of going from unstructured, messy data to knowledge and actionable insights. Provides a broad overview of several topics including statistical data analysis, basic data mining and machine learning algorithms, large-scale data management, cloud computing, and information visualization. Prereqs: (CMSC 141 OR CMSC 131 OR INST 326) AND (MATH 141)
\item \textbf{Using Big Data to Solve Economic and Social Problems (ECON 354)} Prereqs: (DATA 250 OR MATH 241 OR MATH 341 OR MATH 240 OR MATH 461 AND MATH 340 OR MATH 140 AND CMSC 142 OR CMSC 132) AND (CMSC 141 OR CMSC 131 OR INST 326)
\item \textbf{Robotics Programming (ENAE 450)} Prereqs: (DATA 250 OR MATH 241 OR MATH 341 OR MATH 240 OR MATH 461 AND MATH 340 OR MATH 140 AND CMSC 142 OR CMSC 132) AND (CMSC 141 OR CMSC 131 OR INST 326)
\item \textbf{Topics in Aerospace Engineering; Introduction to Autonomous Multi-Robot Swarms (ENAE 488O)} Prereqs: (DATA 250 OR MATH 241 OR MATH 341 OR MATH 240 OR MATH 461 AND MATH 340 OR MATH 140 AND CMSC 142 OR CMSC 132) AND (CMSC 141 OR CMSC 131 OR INST 326)
\item \textbf{Robotics Project Laboratory (ENEE 467)} Prereqs: (DATA 250 OR MATH 241 OR MATH 341 OR MATH 240 OR MATH 461 AND MATH 340 OR MATH 140 AND CMSC 142 OR CMSC 132) AND (CMSC 141 OR CMSC 131 OR INST 326)
\item \textbf{Machine Learning for Materials Science (ENMA 437)} Prereqs: (DATA 250 OR MATH 241 OR MATH 341 OR MATH 240 OR MATH 461 AND MATH 340 OR MATH 140 AND CMSC 142 OR CMSC 132) AND (CMSC 141 OR CMSC 131 OR INST 326)
\item \textbf{Assistive Robotics (ENME  444)}
\item \textbf{Assistive Robotics (ENME 444)} Prereqs: (DATA 250 OR MATH 241 OR MATH 341 OR MATH 240 OR MATH 461 AND MATH 340 OR MATH 140 AND CMSC 142 OR CMSC 132) AND (CMSC 141 OR CMSC 131 OR INST 326)
\item \textbf{Introduction to Robotics (ENME 480)}: This course is an introductory course to the robotics minor and educates students in the elementary concepts of robotics. The topics covered in the course include mathematics of rigid motion, rotations, translations, homogeneous transformations, forward kinematics, inverse kinematics, velocity kinematics, geometric Jacobian, analytical Jacobian, motion planning, trajectory generation, independent joint control, linear control methods such as PD, PID, actuator dynamics, feedforward control for trajectory tracking, force control, basic computer vision concepts including thresholding, image segmentation, and camera calibration. This course also includes a laboratory component to be conducted in the Robot Realization Laboratory in the Engineering Annex Building. Prereqs: (DATA 250 OR MATH 241 OR MATH 341 OR MATH 240 OR MATH 461 AND MATH 340 OR MATH 140 AND CMSC 142 OR CMSC 132) AND (CMSC 141 OR CMSC 131 OR INST 326)
\item \textbf{Special Topics in Geography; Introduction to Spatial Artificial Intelligence (GEOG 398E)}: An introductory course to spatial artificial intelligence (AI), providing a big picture of spatial AI applications (e.g., Google Maps, Uber/Lyft, Earth observation, smart cities, autonomous vehicles), techniques, platforms, trends, debates, etc. The course will cover basics of AI, identify challenges faced by AI techniques in the context of spatial data and applications, and introduce spatial-aware AI methods to address them. AI topics include but are not limited to: spatial data models and knowledge representation, pattern mining, machine learning, perception, planning, etc. Students are expected to have a broad understanding of spatial A concepts, develop intuitions and insights to AI techniques, and have some hands-on experience (Python) at the end of the course. Prereqs: (DATA 250 OR MATH 241 OR MATH 341 OR MATH 240 OR MATH 461 AND MATH 340 OR MATH 140 AND CMSC 142 OR CMSC 132) AND (CMSC 141 OR CMSC 131 OR INST 326)
\item \textbf{Special Topics in Geography; Introduction to Spatial Artificial Intelligence (GEOG398E)} Prereqs: (DATA 250 OR MATH 241 OR MATH 341 OR MATH 240 OR MATH 461 AND MATH 340 OR MATH 140 AND CMSC 142 OR CMSC 132) AND (CMSC 141 OR CMSC 131 OR INST 326)
\item \textbf{Machine Learning for Computational Earth Observation Science (GEOG461)} Prereqs: (DATA 250 OR MATH 241 OR MATH 341 OR MATH 240 OR MATH 461 AND MATH 340 OR MATH 140 AND CMSC 142 OR CMSC 132) AND (CMSC 141 OR CMSC 131 OR INST 326)
\item \textbf{Special Topics in Immersive Media; Creative Experiments with AI (IMDM498E)} Prereqs: (DATA 250 OR MATH 241 OR MATH 341 OR MATH 240 OR MATH 461 AND MATH 340 OR MATH 140 AND CMSC 142 OR CMSC 132) AND (CMSC 141 OR CMSC 131 OR INST 326)
\item \textbf{Designing Fair Systems (INST 204)}: Reviews how specific values are built into different automated decision-making systems as an inevitable result of constructing mechanisms meant to produce specific outcomes. These values create differential outcomes for the different people enmeshed in these systems, but both these values and these systems can be changed to support different values and different outcomes. The class serves as an introduction to the emerging field of algorithmic bias that bridges the disciplines of information science, computer science, law, policy, philosophy, sociology, urban planning, and others. Prereqs: () AND ()
\item \textbf{Object-Oriented Programming for Information Science (INST 326)}: An introduction to programming, emphasizing understanding and
implementation of applications using object-oriented techniques. Topics to
be covered include program design and testing as well as implementation of
programs. Prereqs: INST126
\item \textbf{Privacy, Security and Ethics for Big Data (INST 366)} Prereqs: (DATA 250 OR MATH 241 OR MATH 341 OR MATH 240 OR MATH 461 AND MATH 340 OR MATH 140 AND CMSC 142 OR CMSC 132) AND (CMSC 141 OR CMSC 131 OR INST 326)
\item \textbf{Data Science Techniques (INST 414)}: An exploration of how to extract insights from large-scale datasets. The course will cover the complete analytical funnel from data extraction and cleaning to data analysis and insights interpretation and visualization. The data analysis component will focus on techniques in both supervised and unsupervised learning to extract information from datasets. Topics will include clustering, classification, and regression techniques. Through homework assignments, a project, exams and in-class activities, students will practice working with these techniques and tools to extract relevant information from structured and unstructured data. Prereqs: (CMSC 141 OR CMSC 131 OR INST 326) AND (MATH 141)
\item \textbf{Design and Human Disability and Aging (INST401 )} Prereqs: (DATA 250 OR MATH 241 OR MATH 341 OR MATH 240 OR MATH 461 AND MATH 340 OR MATH 140 AND CMSC 142 OR CMSC 132) AND (CMSC 141 OR CMSC 131 OR INST 326)
\item \textbf{Emerging Technologies and Risk Management (INST461)} Prereqs: (DATA 250 OR MATH 241 OR MATH 341 OR MATH 240 OR MATH 461 AND MATH 340 OR MATH 140 AND CMSC 142 OR CMSC 132) AND (CMSC 141 OR CMSC 131 OR INST 326)
\item \textbf{Introductory Linguistics (LING 200)}: An exploration of the nature of human language. Introduction to the basic concepts and methodology of modern linguistic analysis (sound systems, word formation, sentence structure). Examination of the factors that contribute to dialect differences and the social implications of language variation. Additional topics may include: semantics, pragmatics, language change, writing systems, typology, language universals, comparison with other communication systems. Prereqs: (DATA 250 OR MATH 241 OR MATH 341 OR MATH 240 OR MATH 461 AND MATH 340 OR MATH 140 AND CMSC 142 OR CMSC 132) AND (CMSC 141 OR CMSC 131 OR INST 326)
\item \textbf{Language and Mind (LING 240)}: The study of language as a cognitive phenomenon. Ways of representing people's knowledge of their native language, ways in which that knowledge is attained naturally by children, and how it is used in speaking and listening. Additional topics may include: animal communication, language and the brain, language and thought. Prereqs: (DATA 250 OR MATH 241 OR MATH 341 OR MATH 240 OR MATH 461 AND MATH 340 OR MATH 140 AND CMSC 142 OR CMSC 132) AND (CMSC 141 OR CMSC 131 OR INST 326)
\item \textbf{Meaning through Language: Why are we so good at it?  (LING 260)}: What is it about us humans and our languages that allows us to communicate in ways unmatched by other animals or powerful AI models? The question is ancient, but recent decades have seen great progress in the cognitive science of language, while expanding the diversity of languages investigated. We know much more about how languages vary, how they develop in children, how they are encoded in the mind, and relate to other domains of cognition. Major developments in statistical computation and research on animal cognition also illuminate what is (not) possible without the particular structure of the human mind. We bring this all to bear on our Big Question: What makes human language special from the viewpoint of meaning? Students will come to understand the major features of language as a vehicle for complex thought and a tool for communication. They will use this understanding in analyzing common semantic patterns and everyday conversational dynamics.
\item \textbf{Syntax I (LING 311)}: Basic concepts, analytical techniques of generative syntax, relation to empirical limits imposed by viewing grammars as representations of a component of human mind. Aspects of current theories. Prereqs: (DATA 250 OR MATH 241 OR MATH 341 OR MATH 240 OR MATH 461 AND MATH 340 OR MATH 140 AND CMSC 142 OR CMSC 132) AND (CMSC 141 OR CMSC 131 OR INST 326)
\item \textbf{Phonetics (LING 320)} Prereqs: (DATA 250 OR MATH 241 OR MATH 341 OR MATH 240 OR MATH 461 AND MATH 340 OR MATH 140 AND CMSC 142 OR CMSC 132) AND (CMSC 141 OR CMSC 131 OR INST 326)
\item \textbf{Phonology I (LING 321)} Prereqs: (DATA 250 OR MATH 241 OR MATH 341 OR MATH 240 OR MATH 461 AND MATH 340 OR MATH 140 AND CMSC 142 OR CMSC 132) AND (CMSC 141 OR CMSC 131 OR INST 326)
\item \textbf{Phonology II (LING 322)} Prereqs: (DATA 250 OR MATH 241 OR MATH 341 OR MATH 240 OR MATH 461 AND MATH 340 OR MATH 140 AND CMSC 142 OR CMSC 132) AND (CMSC 141 OR CMSC 131 OR INST 326)
\item \textbf{Historical Linguistics (LING 330)} Prereqs: (DATA 250 OR MATH 241 OR MATH 341 OR MATH 240 OR MATH 461 AND MATH 340 OR MATH 140 AND CMSC 142 OR CMSC 132) AND (CMSC 141 OR CMSC 131 OR INST 326)
\item \textbf{Grammar and Meaning (LING 410)}: The basic notions of semantic theory: reference, quantification, scope relations, compositionality, thematic relations, tense and time, etc. The role these notions play in grammars of natural languages. Properties of logical form and relationship with syntax. Prereqs: (DATA 250 OR MATH 241 OR MATH 341 OR MATH 240 OR MATH 461 AND MATH 340 OR MATH 140 AND CMSC 142 OR CMSC 132) AND (CMSC 141 OR CMSC 131 OR INST 326)
\item \textbf{Calculus I (MATH 140)}: Introduction to calculus, including functions, limits, continuity, derivatives and applications of the derivative, sketching of graphs of functions, introduction to definite and indefinite integrals, and calculation of area. The course is especially recommended for science and mathematics majors. Credit will be granted for only one of the following: MATH 140 or MATH 136 or MATH 120.
\item \textbf{Calculus II (MATH 141)}: Continuation of MATH140, including techniques of integration, improper integrals, applications of integration (such as volumes, work, arc length, moments), inverse functions, exponential and logarithmic functions, sequences and series. Prereqs: MATH140
\item \textbf{Introduction to Linear Algebra (MATH 240)}: Basic concepts of linear algebra: vector spaces, applications to line and plane geometry, linear equations and matrices, similar matrices, linear transformations, eigenvalues, determinants and quadratic forms. All sections of the course will use the software system MATLAB. Credit will be granted for only one of the following: MATH 240 or MATH 341 or MATH 461. Prereqs: MATH 141
\item \textbf{Calculus III (MATH 241)}: An introduction to multivariable calculus, including vectors and vector-valued functions, partial derivatives and applications of partial derivatives (such as tangent planes and Lagrange multipliers), multiple integrals, volume, surface area, and the classical theorems of Green, Stokes and Gauss. All sections of the course will use the software package MATLAB. Credit will be granted for only one of the following: MATH 241 or MATH 340. Prereqs: MATH 340 OR MATH 140
\item \textbf{Multivariable Calculus, Linear Algebra and Differential Equations I (Honors) (MATH 340)}: This is the first semester of the two-semester honors sequence Math 340-341 which gives a unified and enriched treatment of multivariable calculus, linear algebra, and ordinary differential equations, with supplementary material from differential geometry, Fourier series and calculus of variations.
\item \textbf{Multivariable Calculus, Linear Algebra and Differential Equations II (MATH 341)}: This course is a continuation of MATH 340. The honors sequence MATH 340-341 covers roughly the same material of MATH 240, 241, and 246, but in greater depth and rigor. This semeseter will begin with remaining material from Multivariable Calculus (MATH 241) on extrema of functions of several variables and Lagrange multipliers. The remainder of the semester, and the bulk of the course, is then devoted to the theory of Ordinary Differential Equations (MATH 246). Prereqs: MATH 340 OR MATH 140
\item \textbf{Linear Algebra for Scientists and Engineers (MATH 461)}: The course provides an introduction to linear algebra and matrix theory. It is intended primarily for engineering students. This course cannot be used toward the upper level math requirements for MATH/STAT majors. Credit will be granted for only one of the following: MATH 240, MATH 341, or MATH 461. Prereqs: MATH 141
\item \textbf{AI and Ethics (PHIL 211)}: An introduction to a major subfield of contemporary Philosophy, namely applied ethics, and the experience of using some major tools in the practice of philosophy more generally, namely, the construction and formal evaluation of arguments, conceptual analysis, the use of thought experiments, and clear, direct and persuasive writing. Learning how to execute the latter will involve an intense iterative process.
The substantive focus of the course will be the ethical evaluation of AI in some of its current and potentially future incarnations. We’ll examine algorithmic opacity, algorithmic bias
and decision-making, autonomous weapons systems, human-robot interaction, and artificial
moral agents, in order to uncover what, if any, ethical issues they give rise to.
\item \textbf{Public Leaders and Active Citizens (PLCY201)} Prereqs: (DATA 250 OR MATH 241 OR MATH 341 OR MATH 240 OR MATH 461 AND MATH 340 OR MATH 140 AND CMSC 142 OR CMSC 132) AND (CMSC 141 OR CMSC 131 OR INST 326)
\item \textbf{Innovation and Social Change: Creating Change for Good (PLCY215)} Prereqs: (DATA 250 OR MATH 241 OR MATH 341 OR MATH 240 OR MATH 461 AND MATH 340 OR MATH 140 AND CMSC 142 OR CMSC 132) AND (CMSC 141 OR CMSC 131 OR INST 326)
\item \textbf{Ethical, Policy and Social Implications for Science and Technology (PLCY240)} Prereqs: (DATA 250 OR MATH 241 OR MATH 341 OR MATH 240 OR MATH 461 AND MATH 340 OR MATH 140 AND CMSC 142 OR CMSC 132) AND (CMSC 141 OR CMSC 131 OR INST 326)
\item \textbf{Human and Animal Intelligence (PSYC431)} Prereqs: (DATA 250 OR MATH 241 OR MATH 341 OR MATH 240 OR MATH 461 AND MATH 340 OR MATH 140 AND CMSC 142 OR CMSC 132) AND (CMSC 141 OR CMSC 131 OR INST 326)
\item \textbf{Understanding Contemporary Social Problems - Frameworks for Critical Thinking and Strategies for Solutions (SOCY105)}
\end{enumerate}