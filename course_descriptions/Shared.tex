
%%%%%%%%%%%%%%%%%%%%%%%%%%%%%%%%%%%%%%%%%%%%%%%%%%%%%%
% DO NOT EDIT THIS FILE, IT IS AUTOMATICALLY GENERATED
% INSTEAD, EDIT THE FILE HERE:
%
% https://docs.google.com/spreadsheets/d/1qRaEKxhyfwjHDWa3burGeqK0zeC00Br8NhJfnMWXJdk/edit?usp=sharing
%
% If you edit this file, your changes will be overwritten from this
% spreadsheet linked above.
%%%%%%%%%%%%%%%%%%%%%%%%%%%%%%%%%%%%%%%%%%%%%%%%%%%%%%%
\begin{enumerate}
\item \textbf{User Modeling and Personalization (INST436)}: This would encompass recommender systems (like Netflix recommending movies) and personalization algorithms (like the ones that drive social media feeds). Such systems are a really interesting mix of AI and big data, and are one of the most impactful types of AI that people encounter every day. Prereqs: (\prefix{} 216) AND (\prefix{} 221 OR CMSC 421)
\item \textbf{The Current AI Moment (\prefix{} 101)}: This course introduces students to the current moment of AI.  It introduces the historical analogs of AI (e.g., the upheaval of mechanize agricultural, assembly lines, and information technologies) and compares and contrasts how AI is different.  It provides a non-technical history of the major AI developments of the last hundred years and introduces to societal, ethical, economic, and technical questions that are a part of the AI moment.
\item \textbf{Introduction to AI and the Law (\prefix{} 102)}: This course will examine AI Law and Regulation from the perspective of US Law and industry self-regulation with some discussion of select relevant international law and voluntary frameworks (i.e., EU AI Act and ISO/UN/OECD). Specific topics of US and international law may include: American Legal Process and Procedure, Administrative Law for AI Regulation, Constitutional Law Relationships to AI, Intellectual Property Law (Copyright, Trademark, Patent, and Trade Secret Law), Negligence/Products Liability Law, Privacy Law and Privacy Torts, Fundamentals of Contract Law, Industry Self-Regulation and International Frameworks. Prereqs: \prefix{} 101
\item \textbf{Introduction to AI and Food (\prefix{} 103)}: This course will examine the role of AI in American Food.  From how chemical synthesis driven by AI can produce new fertilizers, insecticides, and other tools that can improve the yield and productivity of crops, to using computer vision to monitor livestock, to harvesting crops with robotic agents, this course will explore the role of AI in food production.  Next, the course will cover the role of AI in food storage, transportation, and its trade on commodities markets.  Finally, the course will end with an overview of AI in food preparation, marketing, and helping individuals making nutritional decisions. Prereqs: \prefix{} 101
\item \textbf{Introduction to AI and Creativity (\prefix{} 104)}: This course will examine the role of AI in creative---particularly visual---pursuits.  Student will learn how to generate and edit images and videos using AI tools and learn the limitations of those tools.  Students will learn how to add additional information to these models and learn how to detect AI-created images and videos.  The final project will be to create a portfolio of artwork assisted by AI tools. Prereqs: \prefix{} 101
\item \textbf{AI and UX (\prefix{} 431)}: The purpose of this course is to teach undergraduate and graduate students how and when to integrate AI-assisted search and validation procedures into their information seeking practices. The course could also be designed to provide UMD with opportunities to collect search results data from low tech student assignments using the AI assisted search engines. Comparing search results year on year could enable researchers to monitor tools as they evolve. The course will also teach students to  compare the information retrieval capabilities of standard search engines to the AI tools and monitor the evolution of both over the semesters. Prereqs: (\prefix{} 216) AND (\prefix{} 221 OR CMSC 421)
\item \textbf{AI and Human Creativity (\prefix{} 432)}: Course will focus on the potential impacts of generative AI systems on human creativity. To do this, students will need to learn about 1) capabilities and limitations of generative AI systems, both currently, and in principle / in the future, and in the context of past work, as well as 2) the human systems (cognitive, social) that produce and govern creative work (broadly construed, including scientific discovery, design, and art and music). Prereqs: (\prefix{} 216) AND (\prefix{} 221 OR CMSC 421)
\item \textbf{Trust, Design, and AI (\prefix{} 433)}: This course will focus on examining how user interfaces are designed for generative AI systems, and how those designs evoke (or fail to evoke) trust in users. Generative AI systems are a main component of our modern society, and at the same time, they are too complex and obscure for users to fully understand. As a result, user interfaces must be designed to make generative AI feel usable, reliable, and trustworthy - whether or not those systems actually are! We’ll examine a variety of user interfaces for generative AI systems, and evaluate their usability, reliability, and trustworthiness. Prereqs: (\prefix{} 216) AND (\prefix{} 221 OR CMSC 421)
\item \textbf{Are Robots Taking our Jobs? (\prefix{} 435)}: Are robots taking our jobs? Are there any jobs even worth taking? What other futures of work might we build? This course examines these questions by focusing on the labor process of computer-supported collaborative work (CSCW) in domains ranging from transportation to software development to sex work, drawing on research and theory from sociology, organizational studies, HCI, and more. Design-oriented students will be encouraged to develop interventions to enhance not just productivity but autonomy and democracy. Research-oriented students will learn to study workplaces and situate shopfloor developments in global political economy. Prereqs: (\prefix{} 216) AND (\prefix{} 221 OR CMSC 421)
\end{enumerate}