
%%%%%%%%%%%%%%%%%%%%%%%%%%%%%%%%%%%%%%%%%%%%%%%%%%%%%%
% DO NOT EDIT THIS FILE, IT IS AUTOMATICALLY GENERATED
% INSTEAD, EDIT THE FILE HERE:
%
% https://docs.google.com/spreadsheets/d/1qRaEKxhyfwjHDWa3burGeqK0zeC00Br8NhJfnMWXJdk/edit?usp=sharing
%
% If you edit this file, your changes will be overwritten from this
% spreadsheet linked above.
%%%%%%%%%%%%%%%%%%%%%%%%%%%%%%%%%%%%%%%%%%%%%%%%%%%%%%%
\begin{enumerate}
\item \textbf{nan (PHIL 211)}: An introduction to a major subfield of contemporary Philosophy, namely applied ethics, and the experience of using some major tools in the practice of philosophy more generally, namely, the construction and formal evaluation of arguments, conceptual analysis, the use of thought experiments, and clear, direct and persuasive writing. Learning how to execute the latter will involve an intense iterative process.
The substantive focus of the course will be the ethical evaluation of AI in some of its current and potentially future incarnations. We’ll examine algorithmic opacity, algorithmic bias
and decision-making, autonomous weapons systems, human-robot interaction, and artificial
moral agents, in order to uncover what, if any, ethical issues they give rise to.
\item \textbf{The Current AI Moment (AI 101)}: This course introduces students to the current moment of AI.  It introduces the historical analogs of AI (e.g., the upheaval of mechanize agricultural, assembly lines, and information technologies) and compares and contrasts how AI is different.  It provides a non-technical history of the major AI developments of the last hundred years and introduces to societal, ethical, economic, and technical questions that are a part of the AI moment.
\end{enumerate}