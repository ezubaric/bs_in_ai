
%%%%%%%%%%%%%%%%%%%%%%%%%%%%%%%%%%%%%%%%%%%%%%%%%%%%%%
% DO NOT EDIT THIS FILE, IT IS AUTOMATICALLY GENERATED
% INSTEAD, EDIT THE FILE HERE:
%
% https://docs.google.com/spreadsheets/d/1qRaEKxhyfwjHDWa3burGeqK0zeC00Br8NhJfnMWXJdk/edit?usp=sharing
%
% If you edit this file, your changes will be overwritten from this
% spreadsheet linked above.
%%%%%%%%%%%%%%%%%%%%%%%%%%%%%%%%%%%%%%%%%%%%%%%%%%%%%%%
\begin{enumerate}
\item \textbf{The Current AI Moment (SHAI 101)}: This course introduces students to the current moment of AI.  It introduces the historical analogs of AI (e.g., the upheaval of mechanize agricultural, assembly lines, and information technologies) and compares and contrasts how AI is different.  It provides a non-technical history of the major AI developments of the last hundred years and introduces to societal, ethical, economic, and technical questions that are a part of the AI moment.
\item \textbf{Introduction to AI and the Law (SHAI 102)}: This course will examine AI Law and Regulation from the perspective of US Law and industry self-regulation with some discussion of select relevant international law and voluntary frameworks (i.e., EU AI Act and ISO/UN/OECD). Specific topics of US and international law may include: American Legal Process and Procedure, Administrative Law for AI Regulation, Constitutional Law Relationships to AI, Intellectual Property Law (Copyright, Trademark, Patent, and Trade Secret Law), Negligence/Products Liability Law, Privacy Law and Privacy Torts, Fundamentals of Contract Law, Industry Self-Regulation and International Frameworks. Prereqs: SHAI 101
\item \textbf{Introduction to AI and Food (SHAI 103)}: This course will examine the role of AI in American Food.  From how chemical synthesis driven by AI can produce new fertilizers, insecticides, and other tools that can improve the yield and productivity of crops, to using computer vision to monitor livestock, to harvesting crops with robotic agents, this course will explore the role of AI in food production.  Next, the course will cover the role of AI in food storage, transportation, and its trade on commodities markets.  Finally, the course will end with an overview of AI in food preparation, marketing, and helping individuals making nutritional decisions. Prereqs: SHAI 101
\item \textbf{Introduction to AI and Creativity (SHAI 104)}: This course will examine the role of AI in creative---particularly visual---pursuits.  Student will learn how to generate and edit images and videos using AI tools and learn the limitations of those tools.  Students will learn how to add additional information to these models and learn how to detect AI-created images and videos.  The final project will be to create a portfolio of artwork assisted by AI tools. Prereqs: SHAI 101
\item \textbf{Classical AI Algorithms (SHAI 221)}: This course introduces the foundation of AI theory and practice.  Students learn how to search over structured representations to find optimal solutions to planning problems, how to represent real-world problems in graph structures or game trees, and how to represent and solve AI problems using first-order logic.  Credit only granted to one of 421 or 221. Prereqs: CMSC 131 OR INST 326 OR CMSC 141
\end{enumerate}