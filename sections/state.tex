

\subsection{Statewide Need}

The establishment of \name{} directly aligns with Governor Wes Moore's Executive Order 01.01.2024.02, titled ``Catalyzing the Responsible and Productive Use of Artificial Intelligence in Maryland State Government''. This order emphasizes the need for the State to "ensure the use of AI in Maryland state government is responsible, ethical, beneficial, and trustworthy" and acknowledges that ``AI is transforming society and work in myriad ways, and the pace of that change will continue to accelerate''. 

By developing a curriculum that emphasizes responsible AI development and ethical considerations, \short{} will prepare technical graduates to contribute to the State's commitment to "fairness and equity" in AI applications. The program's focus on innovation and human-centered design will support the State's goal to "explore ways AI can be leveraged to improve State services and resident outcomes''---one of the many interdisciplinary applications of \ai{} possible with \short{}. Furthermore, by educating students on privacy, safety, security, and the importance of transparency in AI systems, \short{} will help build a workforce capable of upholding the principles outlined in the Executive Order, ensuring that AI technologies are implemented in ways that are ``valid and reliable'' and that ``preserve individuals' privacy rights''. 

In summary, \short{} will play a pivotal role in fulfilling the directives of the Executive Order by cultivating a skilled workforce dedicated to the responsible and innovative use of AI, thereby supporting Maryland's mission to "Leave No One Behind" in the rapidly evolving technological landscape. 


\subsection{Demand Analysis}

The establishment of \name{} is economically justified by the increasing market demand for artificial intelligence (AI) professionals, particularly in the Washington, D.C. region. While computer science (CS) provides a broad foundation in computing principles, AI focuses on creating intelligent systems capable of tasks that typically require human intelligence, such as learning, reasoning, and problem-solving. This specialization aligns with the distinct skill sets employers want in creating training data for models, evaluating AI models, and adapting existing AI models.

The white paper "From West to the Rest: AI Workforce Trends in the United States," authored by University of Maryland scholars in the Smith School, provides critical insights into the rapidly expanding AI job market. The report highlights that
\begin{quote}
Stripping out the effects of sheer size, AI Jobs Intensity (ratio of AI to all job postings) yields a different picture. Compared to the aggregated US-level AI Jobs Intensity of 0.56\%, Washington DC ranks \#1 at 1.75\%, followed by VA at 1.36\%, with MD not too far behind at 0.83\% (p. 5).
\end{quote}
This is attributed to the region's strong emphasis on defense, public sector applications, and private industries increasingly incorporating AI into their operations. Moreover, the paper notes that 
\begin{quote}
Driving this growth is an all-out embrace of AI by various agencies of the U.S. federal government, including the Department of Defense. As a direct correlate, many of the major equipment, software, and services suppliers to federal agencies and DoD are based in the MD-DC-VA region. These include, among others, Northrop Grumman, Lockheed Martin, Huntington Ingalls, Booz Allen Hamilton, Accenture, and Deloitte. The region is also home to Amazon HQ2 and Capitol One’s corporate HQ (p. 15).\end{quote}
By situating \short{} in this unique economic landscape, the University of Maryland can address a clear market need. The program will prepare graduates with the specialized skills---natural language processing, generative \ai{}, and how to train and adapt models---described in the report, positioning them to thrive in industries that are rapidly adopting AI technologies. This alignment with workforce demands underscores the strategic importance of \short{} in supporting the D.C. region's leadership in AI innovation and implementation.

% \subsection{Similar Programs in the State}


% \subsection{Impact on Historically Black Institutions}