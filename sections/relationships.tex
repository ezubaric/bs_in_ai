
\subsection{Relationship to AIM}

In Spring 2024, UMD launched the Artificial Intelligence Interdisciplinary Institute at Maryland (AIM), bringing together AI experts across campus to focus on responsible, ethical development and use of the technology to advance public good in industry, government and society. Given the rapid pace of AI development, a core part of AIM’s mission is to reimagine learning in the face of these drastic changes through the introduction of new interdisciplinary programs, including Bachelor of Science and Bachelor of Arts degrees in Artificial Intelligence. Students across all majors will learn the principles of AI and how they apply to their field of study.

As part of this initiative, \short{} is designed to be an interdisciplinary program from the
ground up and thus will use faculty with joint appointments with
\aim{} and homes for courses to create the new interdisciplinary
courses required for this program.


\subsection{Relationship to Computer Science}

This section outlines how \short{} would be different from the existing computer science degree (with which is shares multiple courses).  The first, consistent with its inclusion in \aim{} is that it's interdisciplinary.  Second, it offers an introduction to \abr{ai} early in students' curriculum.  Finally, it offers another option to help cope with the large demand for computational majors at the University of Maryland.

\paragraph{Interdisciplinary}

At present, the computer science degree (where most students learn the foundations of artificial intelligence) while highly technical, does not require courses in ethics, social issues, or in the foundation of intelligence.
%
Because Maryland is a world leader in these areas and because \abr{ai}'s broad impact on society, future leaders in \abr{ai} will need these softer skills to build effective systems that will benefit society.

Moreover, the interdisciplinary courses offered by \short{} will serve as opportunities to bolster the interdisciplinary research that is a fundamental goal of \aim{}: the coursework and projects that begin in this program can lead to startups, research projects, and synergies across the campus.

\paragraph{Early Onramp}

While there is already an machine learning \abr{ml} concentration within the computer science degree, \abr{ai} courses are locked behind long prerequisite chains (the first course in the current \abr{ml} degree is CMSC 320, and the other \abr{ai}-relevant courses are all 400 level).  
%
By integrating \abr{ai} concepts earlier in the curriculum (starting in the introductory seminar and programming courses), students will be able to see the relevance of \abr{ai} to their interdisciplinary interests.

\paragraph{Complementing Existing Computational Majors}

There is a large demand for computational majors at the University of Maryland.  
%
While computer science and information studies are the largest, there are multiple others.
%
Given the pervasiveness of \abr{ai}, this offers another route: one that offers a rigorous technical component with interdisciplinary connections.\footnote{Of course, there are motivated students that create their own programs that are interdisciplinary (e.g., by double majoring or selecting a minor).  By formalizing these processes, we will both make it easier for more students to build this interdisciplinary path and provide students credentials.}
%
The goal is that having an alternate route will be able to attract students from diverse backgrounds and prepare them for jobs in \abr{ai}. 

\subsection{Other Departments}

Supporting correspondence attached.

\subsection{Accreditation}

Following the accepted practice of top computer science programs, we do not intend to pursue accreditation.

\subsection{Cooperative Arrangements}

We plan to pursue an articulation agreement with at least one local community college to ensure students can begin the AI program at a community college and finish it in College Park.