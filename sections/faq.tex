
\question{Does DATA 250 provide enough of a foundation to satisfy both the linear algebra and discrete math requirements?}



\question{Will this be an \abr{lep}?}

Yes, the intention is to submit as an \abr{lep}, although this approval is in parallel to the formal program approval track.

\question{Why is there no statistics course?}

The committee identified three core skillsets from statistics which are already satisfied in other courses:
\begin{itemize}
    \item Regression: In multiple courses, most notably CMSC 320
    \item Statistical Tests: In the AI Measurement Course
    \item Inference: In the AI Foundation Course
\end{itemize}

\question{What is the relationship between CMSC 221 and CMSC 421?  Are they duplicates?}

In the current proposal, we have currently split out the GOFAI course into a higher level one and a lower level one meant for people to take early on in the BS in AI program or as part of some other major.  This is analogous to how Math has a Linear Algebra course for non-majors and one for majors.

How this would actually be implemented is another question.  Perhaps shared lectures and different recitations and homeworks?  Or two completely different courses?  Bill Regli is revampling CMSC 421, and perhaps this could be part of a larger conversation.

\question{Many of these upper level courses are likely of interest to existing students.  Will the classes be cross-listed.}

All courses will be crosslisted in relevant departments.  In almost all cases, this will include CS.  We use the AI prefix in this proposal to better distinguish new from existing courses.


\question{Why use Python?  What about CMSC 131, which uses Java?}

The short version is that we want to be ecumenical about the path that students take to get into the program, and it is fine to show that you can program in either Java (via 131) or Python (preferred), feeding into our 142 course that teaches data structures.

There are several reasons to choose Python: its breadth of packages, its prevalence in the AI arena, and its popularity.  While each of these reasons are intertwined (e.g., its package structure is rich because it is popular), we will address each of these in turn.

But first, Python has its detractors: syntactic whitespace is odd, the lack of typing teaches bad habits, and an interpreted language does not encourage reasoning about the language.  These are valid criticisms and why we are following David van Horn's innovation of starting not with Python but with Pyret (pronounced like ``Pirate''), a language that is similar to Python but that: it explicitly ends code blocks with a keyword \texttt{end}, has better support for documentation and unit tests, and requires types when defining function arguments.

\textbf{Packages:} Common packages like \texttt{pandas} and \texttt{numpy} are designed for Python, and form the backbone of modern data science at \ai{} pipelines.

\textbf{Prevalaence in AI}: Most notably, Pytorch is the most common toolkit for creating new \ai{} models.  As the name implies, it is designed for Python.  

\textbf{Popularity:} \textsc{ieee}\footnote{\url{https://spectrum.ieee.org/top-programming-languages-2024}} reports that Python remains the top ``complete'' programming language in both popularity and in the number of advertised jobs.

\question{What about articulation with local community colleges?}

This is still a work in progress, but the relevant courses at MCCC that directly align with many of our early courses are:\footnote{These are MCCC course codes, not UMD.}

\begin{center}
\begin{tabular}{clr}
\toprule
MCCC Course & Subject & Skills \\
\midrule
CMSC 140 & Programming I & Data Types \\
CMSC 141 & Programming II & Control Flow \\
CMSC 203 & Computer Science I & OOP \\
CMSC 204 & Computer Science II & Data Structures \\
CMSC 206 & Python Programming & Python Syntax \\
MATH 181 & Calculus I & Derivatives \\
MATH 207 & Discrete Math & Proofs \\
MATH 284 & Linear Algebra & Matrix operations \\
\bottomrule
\end{tabular}
\end{center}

\question{What topics are covered in the \bscourse{216} course?}

We plan to roughly follow the outline of \href{https://minitorch.github.io/}{this course} from Sasha Rush but focus more deeply on the lower-level implementation.

\begin{itemize}
    \item The memory hierarchy and latencies 
    \item Allocating and deallocating memory
    \item How basic numeric types are stored in memory
    \item Basic operators (bit-level negation, comparison, addition, etc.)
    \item Floating point operations
    \item Computation graphs
    \item Autodifferentiation
    \item Tensors and their layout in memory
    \item Efficient matrix multiplication
\end{itemize}