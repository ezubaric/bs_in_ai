
\question{Does DATA 250 provide enough of a foundation to satisfy both the linear algebra and discrete math requirements?}

This course is part of the Data Science minor, and it is based on two pre-existing courses: CMSC 250 and
Math 240. One difference is that this is one course instead of two. This course will focus on discrete
mathematics and linear algebra that has applications to data science. 

There are really two questions here.  
\begin{itemize}
    \item \textbf{Does it have enough linear algebra?}  For that, the answer is fairly clearly yes.  While it's good to know what a nullity is and being able to do elimination, it's not strictly necessary for the downstream applications in \ai{}.
    \item \textbf{Does it have enough discrete math?}  Here there's a difference of opinion on the committee.  The consensus however is to follow the lead of the data science program (joint between CS and Math) and to accept the less thorough coverage of combinatorics of DATA 250, as it will cover the graph and proof elements of our more traditional discrete math course.  Of course students can always take the CS version, but this would be an additional overall course they'd have to take.
\end{itemize}


\question{Will this be an \abr{lep}?}

Yes, the intention is to submit as an \abr{lep}, although this approval is in parallel to the formal program approval track.

\question{Why is there no statistics course?}

The committee identified three core skillsets from statistics which are already satisfied in other courses:
\begin{itemize}
    \item Regression: In multiple courses, most notably CMSC 320
    \item Statistical Tests: In the AI Measurement Course
    \item Inference: In the AI Foundation Course
\end{itemize}

\question{What is the relationship between \prefix{} 221 and CMSC 421?  Are they duplicates?}

In the current proposal, we have currently split out the artificial intelligence foundations course into a higher level one and a lower level one meant for people to take early on in the BS in AI program or as part of some other major.  This is analogous to how Math has a Linear Algebra course for non-majors and one for majors.

How this would actually be implemented is another question.  Perhaps shared lectures and different recitations and homeworks?  Or two completely different courses?  Bill Regli is revampling CMSC 421, and perhaps this could be part of a larger conversation.

\question{Many of these upper level courses are likely of interest to existing students.  Will the classes be cross-listed.}

All courses will be crosslisted in relevant departments.  In almost all cases, this will include CS and for the lower-level courses, the BA in AI (HCAI).  We use the \prefix{} prefix in this proposal to better distinguish new from existing courses.


\question{Why use Python?  What about CMSC 131, which uses Java?}

The short version is that we want to be ecumenical about the path that students take to get into the program, and it is fine to show that you can program in either Java (via 131) or Python (preferred), feeding into our 142 course that teaches data structures.

There are several reasons to choose Python: its breadth of packages, its prevalence in the AI arena, and its popularity.  While each of these reasons are intertwined (e.g., its package structure is rich because it is popular), we will address each of these in turn.

But first, Python has its detractors: syntactic whitespace is odd, the lack of typing teaches bad habits, and an interpreted language does not encourage reasoning about the language.  These are valid criticisms and why we are following David van Horn's innovation of starting not with Python but with Pyret (pronounced like ``Pirate''), a language that is similar to Python but that: it explicitly ends code blocks with a keyword \texttt{end}, has better support for documentation and unit tests, and requires types when defining function arguments.

\textbf{Packages:} Common packages like \texttt{pandas} and \texttt{numpy} are designed for Python, and form the backbone of modern data science at \ai{} pipelines.

\textbf{Prevalaence in AI}: Most notably, Pytorch is the most common toolkit for creating new \ai{} models.  As the name implies, it is designed for Python.  

\textbf{Popularity:} \textsc{ieee}\footnote{\url{https://spectrum.ieee.org/top-programming-languages-2024}} reports that Python remains the top ``complete'' programming language in both popularity and in the number of advertised jobs.

\question{What about articulation with local community colleges?}

We plan to pursue an articulation agreement with a local community college to ensure students can begin the AI program at a community college and finish it in College Park.

This is still a work in progress, but the relevant courses at MCCC that directly align with many of our early courses are:\footnote{These are MCCC course codes, not UMD.}

\begin{center}
\begin{tabular}{clr}
\toprule
MCCC Course & Subject & Skills \\
\midrule
CMSC 140 & Programming I & Data Types \\
CMSC 141 & Programming II & Control Flow \\
CMSC 203 & Computer Science I & OOP \\
CMSC 204 & Computer Science II & Data Structures \\
CMSC 206 & Python Programming & Python Syntax \\
MATH 181 & Calculus I & Derivatives \\
MATH 207 & Discrete Math & Proofs \\
MATH 284 & Linear Algebra & Matrix operations \\
\bottomrule
\end{tabular}
\end{center}

A major goal for the Spring is to work on an articulation agreement with a local college (probably MCCC) as part of the MHEC new program requirements.

\question{How will this impact upper-level CS elective enrollments?}

This is complicated and impossible to predict with precision, but it might be helpful to break this into several categories.

\begin{itemize}
    \item \textbf{Lower:} For some courses, the material will be covered in new lower-level courses earlier, so only the really interested students would take upper level \ai{} courses (Traditional AI and Neural Networks).  So those courses should see lower enrollments.
    \item \textbf{Uncertain:} For other courses, there are multiple factors at play.  While the increased number of students might increase enrollments (although this is complicated by the CS LEP being more stringent), we are offering many more electives that should distribute the load.  But given this is based on popularity, some courses might see increase enrollments.
    \item \textbf{Higher:} Despite that, for some courses that are now required within a specialization, the enrollments are likely going to be higher (e.g., Natural Language Processing).
\end{itemize}

However, one of the primary reasons for the new major is that to hiring more TTK faculty to teach these upper level courses.


\question{What topics are covered in the \bscourse{216} course?}

We plan to roughly follow the outline of \href{https://minitorch.github.io/}{this course} from Sasha Rush but focus more deeply on the lower-level implementation.

\begin{itemize}
    \item The memory hierarchy and latencies 
    \item Allocating and deallocating memory
    \item How basic numeric types are stored in memory
    \item Basic operators (bit-level negation, comparison, addition, etc.)
    \item Floating point operations
    \item Computation graphs
    \item Autodifferentiation
    \item Tensors and their layout in memory
    \item Efficient matrix multiplication
\end{itemize}

\question{These are an odd mix of topics for a course, why do they belong together?}

The goal of this course is twofold.  First, students should understand the nuts and bolts of how these models are trained.  Both at the software level---the abstractions that instantiate abstract functions and generate derivatives from them---and the hardware level---how the operations are carried out on the \abr{gpu}.  But beyond that, students should understand the optimization of these models, which requires other skill sets.  Fortunately, this helps them better understand computer systems.

There are two ways to optimize a model: make it faster or make it smaller (without sacrificing information).  Making it smaller requires using fewer bits per parameters, which requires understand computer representations of floating point numbers (and the disadvantages of quantization).  Making things faster requires understanding the memory layout of GPUs and how parallel computations can operate on the layout of matrices in GPU memory.

\question{This is a fairly new course without an obvious textbook, why should we have something so cutting edge as a low-level undergraduate course.}

