
\subsection{Mission and Purpose}

The goal of \name{} is to train the next generation of innovators and builders of artificial intelligence technology to apply their expertise to help solve interdisciplinary problems in science and society.

\subsection{Educational Objectives}

Currently at UMD, AI is often taught as an advanced subject requiring substantial prerequisites. In addition, \ai{}-related coursework in the Computer Science department is significantly over-enrolled every semester, as are many of the AI and machine learning courses across other departments, including Math, the College of Information, Physics, Engineering, and others. 
%
To address this need---that reflects how \ai{} is becoming a part of many disciplines across the university from analyzing polling data in Journalism to detecting irrigation issues from satellite imagery in Agriculture---the new \bsai{} will build on Maryland's existing strengths to  offer an integrated curriculum to expose students to AI starting with their introductory courses and then allowing them to gain depth and expertise within AI or building breadth by connecting AI across the campus. 

In addition to serving these communities, the introductory courses---foundations of \ai{}, computation graphs, and \ai{}-first introductions to programming---will offer exposure to both introductory programming and core competencies of working with and understanding AI even for students outside the key major.  It will thus make AI learning accessible to a much larger student population in a curricular area that is in high demand.

Students with a BS in AI will be prepared for a variety of career paths, including AI developer, prompt engineer, dataset curation, data scientist, and AI-assisted content creation.  These AI skills will also be useful in application areas such as natural language processing, computer vision, data science and analytics, user experience and design, robotics, business analytics, and geospatial reasoning. Students can also use their technical expertise to enter careers in policy, law, education, agriculture, or healthcare. Finally, our program will prepare students who want to continue their education and pursue an MS or PhD.


\subsection{Selective Admissions}

\name{} will be designated as a Limited Enrollment Program (LEP) at the University of Maryland, adhering to specific admission criteria to manage enrollment and maintain academic standards.

\subsubsection*{Freshman Admission}
Prospective students applying as freshmen to the University of Maryland can indicate their interest in \name{} on their general application. Admission to \short{} is competitive and based on a holistic review of the applicant's academic record, co-curricular involvement, relevant research experience or personal projects, and any honors or awards. Meeting the minimum requirements does not guarantee admission.

\subsubsection*{Transfer Admission}
Transfer students from other institutions must complete the following Gateway courses with a minimum grade of B- before applying:
\begin{itemize}
    \item \textbf{CMSC141} (Programming with Purpose I: Data-Centric Computing) or \textbf{CMSC131} (Object-Oriented Programming I) or \textbf{CMSC133} (Object-Oriented Programming I Beyond Fundamentals) or \textbf{INST 326} (Object-Oriented Programming)
    \item \textbf{CMSC142} (Programming with Purpose II: Data Structures and Algorithms) or \textbf{CMSC132} (Object-Oriented Programming II)
    \item \textbf{MATH140} (Calculus I) 
\end{itemize}

Additionally, a minimum cumulative GPA of 3.0 in all courses taken at the University of Maryland and other institutions is required. Transfer applicants undergo a selective review process considering academic performance, co-curricular activities, computing-related research experience or personal projects, and college-level honors and awards. Completion of Gateway requirements does not guarantee admission.

\subsubsection*{Internal Transfer (Current UMD Students)}
Current University of Maryland students wishing to change their major to \short{} or add it as a second major must:
\begin{enumerate}
    \item Successfully complete the Gateway requirements listed above.
    \item Apply through the LEP application available after the add/drop deadline each semester.
\end{enumerate}

Admission is selective, and meeting the Gateway criteria does not ensure acceptance. Students should be prepared to select an alternate major if not admitted to \short{}.

\subsubsection*{LEP Policies}
\begin{itemize}
    \item Only one Gateway or performance review course may be repeated to earn the required grade, and it may only be repeated once. A grade of ``W'' is considered an attempt.
    \item Students may apply only once to an LEP.
    \item Students directly admitted who fail to meet performance review criteria will be dismissed from the major and may not reapply.
    \item A minimum cumulative GPA of 2.00 must be maintained after admission; failure to do so will result in dismissal.
    \item Appeals can be made in writing to the College of Computer, Mathematical, and Natural Sciences advising office.
\end{itemize}

These policies ensure that \short{} maintains high academic standards and provides quality education to its students.

\subsection{Factors Considered in Developing the Proposed Curriculum}

\name{} is needed to address increasing demand in \ai{} skills and to ensure that those skills are taught early enough in an undergraduate curriculum so that they can be applied across disciplines.

\subsection{Recruiting and Retaining a Diverse Student Body}

\name{} is committed to fostering a diverse and inclusive environment by integrating and expanding upon existing initiatives within the University of Maryland's computing and information sciences communities. Building upon the successful programs of the Iribe Initiative for Inclusion and Diversity in Computing (I4C), \short{} will actively participate in and promote activities such as the Mentoring Program, Conference Scholarships, and the Tech + Research initiative (many of the projects there are already \ai{}-centered). These programs have a proven track record of supporting underrepresented students in computing fields by providing mentorship, professional development opportunities, and research experiences. By aligning with these initiatives, \short{} aims to create a supportive network that encourages the recruitment and retention of a diverse student body. 

Furthermore, \short{} will collaborate with diversity and inclusion efforts to ensure that the program's culture reflects a commitment to accessibility and the democratization of information. This collaboration includes participating in outreach programs targeting K-12 students, such as the TRAILS AI Summer Academy, which introduces high school students to artificial intelligence concepts and applications. By engaging with these outreach efforts, \short{} seeks to inspire a diverse pipeline of future students who are well-prepared and motivated to pursue studies in artificial intelligence. 

By leveraging and contributing to these established diversity and inclusion programs, \short{} will create an environment that not only attracts a diverse group of students but also provides the necessary support and resources to ensure their success throughout their academic journey.
