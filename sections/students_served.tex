Currently at UMD, AI is often taught as an advanced subject requiring substantial prerequisites. In addition, AI-related coursework in the Computer Science department is significantly over-enrolled every semester, as are many of the AI and machine learning courses across other departments, including Math, the College of Information, Physics, Engineering, and others. This reflects how AI is becoming a part of many disciplines across the university from analyzing polling data in Journalism to detecting irrigation issues from satellite imagery in Agriculture.  Because the technical foundations of AI have traditionally only been in computer science, the BS in AI will fulfill the twin goals of both improving the reach of AI and taking pressure off of traditionally over-enrolled courses.  The BS in AI will offer an integrated curriculum to students that exposes them to AI learning through introductory level courses and then builds on that foundation through more advanced courses. 

In addition to serving these communities, the introductory courses will offer exposure to both introductory programming and core competencies of working with and understanding AI even for students outside the key major.  It will thus make AI learning accessible to a much larger student population in a curricular area that is in high demand.

Students with a BS in AI will be prepared for a variety of career paths, including AI developer, prompt engineer, dataset curation, data scientist, and AI-assisted content creation.  These AI skills will also be useful in application areas such as natural language processing, computer vision, data science and analytics, user experience and design, robotics, business analytics, and geospatial reasoning. Students can also use their technical expertise to enter careers in policy, law, education, agriculture, or healthcare. Finally, our program will prepare students who want to continue their education and pursue an MS or PhD.
